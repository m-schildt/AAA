%%%%%%%%%%%%%%%%%%%%%%%%%%%%%%%%%%%%%%%%%
% Masters/Doctoral Thesis 
% LaTeX Template
% Version 2.5 (27/8/17)
%
% This template was downloaded from:
% http://www.LaTeXTemplates.com
%
% Version 2.x major modifications by:
% Vel (vel@latextemplates.com)
%
% This template is based on a template by:
% Steve Gunn (http://users.ecs.soton.ac.uk/srg/softwaretools/document/templates/)
% Sunil Patel (http://www.sunilpatel.co.uk/thesis-template/)
%
% Template license:
% CC BY-NC-SA 3.0 (http://creativecommons.org/licenses/by-nc-sa/3.0/)
%
%%%%%%%%%%%%%%%%%%%%%%%%%%%%%%%%%%%%%%%%%

%----------------------------------------------------------------------------------------
%	PACKAGES AND OTHER DOCUMENT CONFIGURATIONS
%----------------------------------------------------------------------------------------

\documentclass[
12pt, % The default document font size, options: 10pt, 11pt, 12pt
oneside, % Two side (alternating margins) for binding by default, uncomment to switch to one side
english, % ngerman for German
onehalfspacing, % Single line spacing, alternatives: onehalfspacing or doublespacing, singlespacing
%draft, % Uncomment to enable draft mode (no pictures, no links, overfull hboxes indicated)
%nolistspacing, % If the document is onehalfspacing or doublespacing, uncomment this to set spacing in lists to single
%liststotoc, % Uncomment to add the list of figures/tables/etc to the table of contents
%toctotoc, % Uncomment to add the main table of contents to the table of contents
%parskip, % Uncomment to add space between paragraphs
%nohyperref, % Uncomment to not load the hyperref package
headsepline, % Uncomment to get a line under the header
%chapterinoneline, % Uncomment to place the chapter title next to the number on one line
%consistentlayout, % Uncomment to change the layout of the declaration, abstract and acknowledgements pages to match the default layout
]{MastersDoctoralThesis} % The class file specifying the document structure

\usepackage[utf8]{inputenc} % Required for inputting international characters
\usepackage[T1]{fontenc} % Output font encoding for international characters

\usepackage{mathpazo} % Use the Palatino font by default

\usepackage[backend=bibtex,style=authoryear,natbib=true]{biblatex} % Use the bibtex backend with the authoryear citation style (which resembles APA)

\addbibresource{library.bib} % The filename of the bibliography

\usepackage[autostyle=true]{csquotes} % Required to generate language-dependent quotes in the bibliography
\usepackage{{booktabs}}
\usepackage{multirow}
%----------------------------------------------------------------------------------------
%	MARGIN SETTINGS
%----------------------------------------------------------------------------------------

\geometry{
	paper=a4paper, % Change to letterpaper for US letter
	inner=2.5cm, % Inner margin
	outer=3.8cm, % Outer margin
	bindingoffset=.5cm, % Binding offset
	top=1.5cm, % Top margin
	bottom=1.5cm, % Bottom margin
	%showframe, % Uncomment to show how the type block is set on the page
}

%----------------------------------------------------------------------------------------
%	THESIS INFORMATION
%----------------------------------------------------------------------------------------

\thesistitle{The effect of government support during the COVID-19 pandemic: Firm-level evidence from Germany} % Your thesis title, this is used in the title and abstract, print it elsewhere with \ttitle
\supervisor{Dr. Simon \textsc{Munzert}} % Your supervisor's name, this is used in the title page, print it elsewhere with \supname
\examiner{} % Your examiner's name, this is not currently used anywhere in the template, print it elsewhere with \examname
\degree{Master of Data Science for Public} % Your degree name, this is used in the title page and abstract, print it elsewhere with \degreename
\author{Marco \textsc{Schildt}} % Your name, this is used in the title page and abstract, print it elsewhere with \authorname
\addresses{} % Your address, this is not currently used anywhere in the template, print it elsewhere with \addressname

\subject{} % Your subject area, this is not currently used anywhere in the template, print it elsewhere with \subjectname
\keywords{} % Keywords for your thesis, this is not currently used anywhere in the template, print it elsewhere with \keywordnames
\university{\href{https://www.hertie-school.org}{Hertie School}} % Your university's name and URL, this is used in the title page and abstract, print it elsewhere with \univname
\department{} % Your department's name and URL, this is used in the title page and abstract, print it elsewhere with \deptname
\group{} % Your research group's name and URL, this is used in the title page, print it elsewhere with \groupname
\faculty{} % Your faculty's name and URL, this is used in the title page and abstract, print it elsewhere with \facname

\AtBeginDocument{
\hypersetup{pdftitle=\ttitle} % Set the PDF's title to your title
\hypersetup{pdfauthor=\authorname} % Set the PDF's author to your name
\hypersetup{pdfkeywords=\keywordnames} % Set the PDF's keywords to your keywords
%\hypersetup{citecolor=black}
%\hypersetup{urlcolor=black}
}

\begin{document}

\frontmatter % Use roman page numbering style (i, ii, iii, iv...) for the pre-content pages

\pagestyle{plain} % Default to the plain heading style until the thesis style is called for the body content

%----------------------------------------------------------------------------------------
%	TITLE PAGE
%----------------------------------------------------------------------------------------

\begin{titlepage}
\begin{center}

\vspace*{.06\textheight}
{\scshape\LARGE \univname\par}\vspace{1.5cm} % University name
\textsc{\Large Master Thesis}\\[0.5cm] % Thesis type

\HRule \\[0.4cm] % Horizontal line
{\huge \bfseries \ttitle\par}\vspace{0.4cm} % Thesis title
\HRule \\[1.5cm] % Horizontal line
 
\begin{minipage}[t]{0.4\textwidth}
\begin{flushleft} \large
\emph{Author:}\\
{\authorname} % Author name - remove the \href bracket to remove the link
\end{flushleft}
\end{minipage}
\begin{minipage}[t]{0.4\textwidth}
\begin{flushright} \large
\emph{Supervisor:} \\
{\supname} % Supervisor name - remove the \href bracket to remove the link  
\end{flushright}
\end{minipage}\\[3cm]
 
\vfill

\large \textit{A thesis submitted in fulfillment of the requirements\\ for the degree of \degreename}\\[0.3cm] % University requirement text
%\textit{in the}\\[0.4cm]
%\groupname\\\deptname\\[2cm] % Research group name and department name
 
\vfill

{\large \today}\\Wordcount: X.XXX\\[4cm] % Date
%\includegraphics{Logo} % University/department logo - uncomment to place it
 
\vfill
\end{center}
\end{titlepage}


\cleardoublepage


%----------------------------------------------------------------------------------------
%	ABSTRACT PAGE
%----------------------------------------------------------------------------------------
\def\abstractname{Executive Summary}
\begin{abstract}

\addchaptertocentry{\abstractname} % Add the abstract to the table of contents
\noindent

By matching information on firm level from two sources this thesis provides insights on the effects of government support during the COVID-19 pandemic. To mitigate the effects of the pandemic shock the German government responded with unprecedented fiscal efforts to provide liquidity to the economy and prevent business from failing. 
To provide an impact assessment a quasi-experimental approach with a DiD regression is employed to estimate the causal effect of aid measures on the liquidity and solvency of companies.
The results suggest that in 2020 loan-based measures supported companies by improving their liquidity by 5.3 \%, while grant based aid was insufficient as a liquidity injection since beneficiaries had no improvement in liquidity and relatively higher debt.
Only in 2021 grants programs became effective in providing liquidity and preserving the equity of firms. In 2021 the liquidity increase of grant beneficiaries was 7.7 \%, twice as high as the effect of grants in 2021.

Moreover, insolvencies of aid beneficiaries were analyzed. Results suggest that they were already significantly weaker before the pandemic and that aid measures were less effective in providing them support compared to the overall population of beneficiaries. However, a sector view reveals that the industries that are most represented amongst beneficiaries (“Food” and “Accommodation”) show insolvency rates well below the average, suggesting that aid measures were successful in reduced the chance of insolvency in their presumed main target sectors.

Finally, to provide a deeper understanding of the aid measures and fully exploit the data granularity a combination of generalized propensity scores and a GAM is employed to advance the assessment from a binary to continuous treatment perspective.
This approach allows to visualize the effect of aid by showing the relationship between aid payments and change liquidity as well as solvency of beneficiaries. Overall, the visualizations are in line with results from the “average” view DiD results and don’t show any concerning abnormalities. In greater detail the visualization reveal heterogeneity amongst beneficiaries from different industries, which [do not] connect with the observed heterogeneity in insolvencies of different industries.


Insolvency

Finally, to provide a deeper understanding of the aid measures and fully exploit the data granularity a combination of generalized propensity scores and a GAM is employed to advance the assessment from a binary to continuous treatment perspective.
This approach allows to visualizes the effect of aid by showing the relationship between aid payments and change liquidity as well as solvency of beneficiaries. Overall, the visualization are in line with results from the “average” view DiD results and don’t show any concering abnormalities. In greater detail the visualization reveal heterogeneity amongst industries, like the risk of insolvencies.



\end{abstract}


%----------------------------------------------------------------------------------------
%	LIST OF CONTENTS/FIGURES/TABLES PAGES
%----------------------------------------------------------------------------------------
\hypersetup{linkcolor=black}
\tableofcontents % Prints the main table of contents

\listoffigures % Prints the list of figures

\listoftables % Prints the list of tables

%----------------------------------------------------------------------------------------
%	ABBREVIATIONS
%----------------------------------------------------------------------------------------

\begin{abbreviations}{ll} % Include a list of abbreviations (a table of two columns)

\textbf{ATT} & \textbf{A}verage \textbf{T}treatment Effect on the \textbf{T}reated\\
\textbf{EU COM} & \textbf{EU}ropean \textbf{COM}mission\\
\textbf{SMEs} & \textbf{S}mall and \textbf{M}edium-sized \textbf{E}nterprises\\




\end{abbreviations}




%----------------------------------------------------------------------------------------
%	THESIS CONTENT - CHAPTERS
%----------------------------------------------------------------------------------------

\mainmatter % Begin numeric (1,2,3...) page numbering

\pagestyle{thesis} % Return the page headers back to the "thesis" style

% Include the chapters of the thesis as separate files from the Chapters folder
% Uncomment the lines as you write the chapters



% Chapter Template

\chapter{Introduction to the government support in Germany} % Main chapter title

\label{Chapter1} % Change X to a consecutive number; for referencing this chapter elsewhere, use \ref{ChapterX}


%----------------------------------------------------------------------------------------
%	No SubsECTION 
%----------------------------------------------------------------------------------------


The Covid-19 pandemic has severely affected the entire world with many devastating consequences. Businesses in many parts of the economy were struggling to survive due to shocks in demand, lockdowns from governments and disrupted supply chains \parencite{eu_com_temporary_2020}.

To sustain the economy and prevent businesses them from bankruptcy during the pandemic, the German government responded with a range of policies. Beside the various measures like labor cost subsidies, temporary changes in the insolvency law and tax reliefs, the financial support through grants and loans was unprecedented. The financial support was available for businesses in all sizes that affected by the pandemic ranging from self-employed individuals to small and medium-sized enterprises (SMEs) up to very large companies. From spring 2020 to summer 2022, grants, loans, recapitalizations and guarantees alone accounted for a total of around EUR 130 billion \parencite{bmwk_uberblickspapier_2022}. 

Grants and loans were the two groups of instruments with widespread use by the German government. Grants are funds provided by the government to businesses that are not needed to be repaid. Grants are usually subject to the terms and conditions, but do not require any consideration in return. Whereas financial support measures based on loans have to be repaid, like standard bank loans. The advantage over a normal credit transaction are beneficial conditions that a company would not have received under normal circumstances from a bank. Especially not in times where the company's future is uncertain and linked to the further development of the pandemic. 

From the companies' point of view, both types of aid have the immediate effect of a liquidity injection, meaning that additional cash is available. However, despite the liquidity additional debt doesn't offset deteriorating effect of potential losses on the equity of firms. Together with the repayment obligation, loans have a contrasting effect on the solvency compared to grants which could only become relevant in the long run.

Irrespective of the concrete form of the aid measures their scale manifests a fiscal effort that is inconceivable at this magnitude under normal conditions. Usually, governments are not permitted to provide extensive subsidies, due to concerns of distorting competing in the European single market \parencite{claici_european_2022}. The permissibility of subsidies is comprehensively regulated by European state aid laws. Before a subsidy is considered permissible under this legal framework an assessment of its necessity, incentive effect, proportionality and effect on trade and competition is needed \parencite{claici_european_2022}. In light of the ongoing pandemic, the EU relaxed rules on subsidies by introducing the Temporary Framework for State aid measures to support the economy in the current COVID-19 outbreak, which provided national governments more freedom to come up with quick and extensive policy responses to support businesses \parencite{eu_com_temporary_2020}.

Under the framework a justification the financial support measures including the necessity, the appropriateness and proportionality to remedy the impact of the pandemic in the economy was required from the German government \parencite{eu_com_temporary_2020}. Defining and deciding on the appropriateness as well as proportionality of support measures is a complex and challenging task. Due to the unpredictable scale of the pandemic, uncertainty is immense. On the other hand, the effect of support measures is nothing trivial to estimate, given that their scale was unprecedented. To ensure that the support measures are effective, efficient, a good understanding of aid instruments is vital. 

A key role of assessments is to provide this understanding. This thesis aims to access whether grants and loans were successful in in providing the much-needed liquidity and preventing businesses from bankruptcy.

In specific, this thesis tries to answer whether the aid measures were successful in providing liquidity, by measuring the causal effect of aid instruments on the firm liquidity. A causal effect that is more than marginal would allow to attribute the liquidity increase to the aid program and bring support for their successfulness. In case of an unmeasurable or marginal effects the injected liquidity could have been fully absorbed by aid recipients or have been generally ineffective. Questions about whether the aid was providing sufficient support 
would arise and, in any case, challenge the successfulness of liquidity support through grants and loans. Considering the magnitude of government intervention and the scale of fiscal efforts aid measures are expected to a measurable effect on firm liquidity.

With regard to the aforementioned differences between grants and loans this thesis will also seek to answer whether these instruments have different effects on the solvency of aid beneficiaries. It is expected that loans increase the leverage, while grants are expected to have the opposite effect and help to preserve the equity of firms.

The analysis of liquidity and solvency will be supplemented by a view on dose response functions to provide a granular picture of the effects at different levels of aid. The insights will be used to answer whether the effects of aid are varying depending on the level of aid. 

And with a sectoral overview the assumption of heterogenous effects of grants between industries with different cost structures from \parencite{bischof_bedeutung_2021} will be tested.

Finally, this paper looks closer at aid beneficiaries that are insolvent by spring 2023 to compare the effectiveness of their aid as well as the heterogeneity assumption.

The approach differs from existing assessments in two regards. First, only publicly available data sources are used. Previous assessment of pandemic aid with a focus on Germany were only conducted on the basis of survey data \parencite{marek_impact_2022,bertschek_german_2022,dorr_small_2022,bischof_bedeutung_2021}. As such, this work demonstrates a way of policy evaluation that is not reliant on surveys, which are typically time and resource intensive.

Second, an approach with generalized propensity scores is adopted, which has so far only played a role on the context of other subsidy assessments \parencite{selebaj_effects_2021,carboni_effect_2017}. This methodology allows the estimation and visualization of dose-response functions to determine whether the causal effects support vary depending on the different size of the aid measures \parencite{selebaj_effects_2021}.

This thesis is organized as follows. After this introduction, chapter \ref{Chapter2} provides an overview of the general effects of COVID-19 pandemic on businesses as well as the already existing impact assessments of government support. Chapter \ref{Chapter3} presents the data sources and chapter \ref{Chapter4} explains the methodologies. In chapter \ref{Chapter5}, the findings about the effect of aid measures are shown. Chapter \ref{Chapter6} addresses the policy implications and concludes. 





\begin{table}
\caption{Overview of support instruments}
\label{tab:treatments}
\centering
\begin{tabular}{lr}
\toprule
                   Beihilfeinstrument &               aid \\
\midrule
Andere Formen der Kapitalintervention &  9,017,729,574.33 \\
                           Bürgschaft &  1,144,042,410.02 \\
              Eigenkapitalinstrumente &  2,419,881,701.00 \\
      Kredite/rückzahlbare Vorschüsse &    753,217,635.27 \\
            Sonstiges (bitte angeben) & 14,244,894,962.69 \\
               Zinsgünstiges Darlehen & 10,500,942,385.00 \\
                         Zinszuschuss &  9,383,307,910.00 \\
                             Zuschuss & 24,959,647,770.34 \\
\bottomrule
\end{tabular}
}

\end{table}
    

%\caption{The effects of treatments X and Y on the four groups studied.}


% Chapter Template

\chapter{Literature Review} % Main chapter title

\label{Chapter2} % Change X to a consecutive number; for referencing this chapter elsewhere, use \ref{ChapterX}

%----------------------------------------------------------------------------------------
%	SECTION 1
%----------------------------------------------------------------------------------------

\section{Pandemic effects}

The negative consequences of the COVID-19 pandemic on the economy have become evident in many areas. Many businesses were severely affected by drops in demand and through lockdowns ordered by authorities. When business operations are halting while costs like rent or personal costs continue occurring the pandemic shock eventually leads to negative cash flows for many firms \parencite{fernandez-cerezo_firm-level_2021}.
Depending on the affectedness of the business, the liquidity reserves will inevitably deteriorate and eventually liquidity shortfalls are inescapable with negative cash flows \parencite{puhr_have_2021}.
Although the demand of liquidity is individual for every company, a simulation study from Italy quantified the total liquidity deficit of all Italian SMEs caused by the Covid-19 shock to 83.7 billion Euros at the end of 2020 \parencite{bellucci_consequences_2022}. 
In comparison for the Belgian corporate sector, in a scenario without policy interventions, the drop in liquidity by September 2020 is quantified with 28.2 billion Euros \parencite{tielens_belgian_2020}. 

Empirical results from Tielens et al. (2020) suggest that in Belgium even businesses that used to be profitable require a large amount of additional financing to offset their liquidity shortfall.

Without a return of profits, firms are in need of liquidity injection to, either through additional equity or via debt. For smaller unlisted firms it is usually not possible to easily raising equity, therefore they usually left with the debt option and rely on credits from banks \parencite{pagano_covid-19_2022}. Pagano and Zechner (2022) analyzed the effects of covid 19 on companies’ financial performance in the EU. Their evidence suggests differences in the effects between large firms and small and medium sized enterprises. By comparing the years 2019 and 2020 the authors found that smaller companies tend to increase their ratio of total debt to total assets (debt-ratio) whereas, large companies also increase their leverage, but significantly less. Regarding liquidity, their findings suggest that small and medium sized enterprises increased their cash-to-total-assets-ratio more than large companies. Small companies did so even more than medium sized ones. However, the authors could only speculate over the reason behind of this observation. Plausible reasons were precautionary cash hording and greater risk aversion. Additionally, the authors raise the theory that smaller companies were able to raise cash more easily due to the claim, that loan guarantee programs favored small firms. However, the analyzed sample of small and medium sized enterprises was not representative of any specific industry, nor of aid recipients.

However, credits from banks, if obtainable, increases the firms leverage and could make the firm vulnerable to new liquidity shortfalls. And additional leverage only prevents from insolvency if there is a prospect that future cash flows will enable a firm to service the additional debt. Regarding the capital structure, increased leverage means a weaker equity ratio. An early Survey study from September 2020 analyzed the implications of the pandemic crisis on the equity of Germany companies and reported that for most companies the equity ratio did not change, however a strong sectoral heterogeneity with travel and gastronomy having a reduction in the equity ratio between 1.8 \% and 1.5 \% \parencite{peichl_eigenkapitalentwicklung_2021}. In Spain a survey looked at the indebtedness as well as the cash ratio of enterprises and reported findings that support the heterogeneity of the covid 19 shock across firms and, that the impact was larger for small, young and less productive firms located in urban areas \parencite{fernandez-cerezo_firm-level_2021}. Further support for the heterogeneity of the impact of the COVID-19 crisis on firms’ sales and costs came from Belgium \parencite{dhyne_belgian_2021}.

A simulation on 14 relatively well-covered European countries estimated that an increase in the financial debt of companies has on average a negative impact on the growth of investment after the crisis, indicating negative long-term effects increased leverage \parencite{demmou_insolvency_2021}.



%----------------------------------------------------------------------------------------
%	SECTION 2
%----------------------------------------------------------------------------------------

\section{Government support effects}

Policy intervention to "prevent” the effects and save businesses for a fast economic recovery.
First assessments were modeling approaches.
Already lots of early assessments of state aid, also at a firm level.
For getting a better understanding on the effect of aid schemes in Germany a paper analyses the effect of a company’s cost structure on the effectives of aid measures (Bischof, Karlsson, Rostam-Aschar, Simon, 2021). 
This paper assumes that companies within the same sector have a similar cost structure. 
Since aid in Germany is based on the cost structure of companies, the authors conclude that based on the generalized approach of aid schemes, the effectiveness of aid is varying between business sectors.
% Chapter Template

\chapter{Data Sources} % Main chapter title

\label{Chapter3} % Change X to a consecutive number; for referencing this chapter elsewhere, use \ref{ChapterX}

%----------------------------------------------------------------------------------------
%	SECTION 1
%----------------------------------------------------------------------------------------

\section{Data on Government support}

This thesis uses data from the European state aid transparency database \parencite{eu_com_state_2023}. 
The database contains information about individual award data like beneficiary name, amount, Date of Granting, and the purpose of the state aid \parencite{eu_com_state_2023}. 

The legal base for the transparency requirement aid payments is Temporary Framework for State aid, however payments under 100.000 EUR (10.000 EUR for an agricultural firm) are exempted from the transparency requirement, insofar as the data base is not comprehensive. Nevertheless, as of spring 2023 for Germany 135.478 cases of aid related to the COVID-19 pandemic were disclosed under the objective “Remedy for a serious disturbance in the economy”. Unfortunately, the disclosed titles of the aid measures and case numbers do not allow for reconciliation to the official names of the aid programs due to amendments and overlaps. In addition, in the case of bigger companies the aid was usually calculated on a group level but awarded and paid in full to just one company of the group. 

Another thing to be mentioned is that most direct grants were granted and paid on a provisional basis and are still subject to a final determination of the granted amount. For the research, this does not pose an issue, since the effects of payments will be observable regardless of whether the amount of aid got adjusted in a later period. 





%----------------------------------------------------------------------------------------
%	SECTION 2
%----------------------------------------------------------------------------------------

\section{Company level financial information}

In Germany, corporations are legally required to disclose their annual financial statements in the Federal Gazette. Although the discloser of financial information is legally required for corporations, and there are various exemptions for example for companies that are consolidated into other companies' balance sheets, and also for non-compliant companies.
The requirement on the disclosed financial information is depending on the size of the company. Bigger companies above certain thresholds additionally need to disclose their profit and loss statements and management reports additionally. However, all companies that are subject to regulation must disclose at least their balance sheet. 
To not exclude SMEs systematically, due to missing profit and loss statements, the data collection process is limited to balance sheet information. 

The main constraint in processing the financial information from the Federal Gazette is the vastly unstandardized formatting of balance sheets causing limited readability in the data parsing step. By scraping and parsing the financial information for beneficiaries of the pandemic government support in total balance sheets of 23.505 companies for at least one year in the period 2018 - 2022 was obtained. Figure \ref{fig:FirmSizes} shows that SMEs are represented in the dataset and that the distribution between 2020 and 2021 is comparable.

It is necessary to mention, that in cases of two companies of the same name the matching is prone to mismatches, and information for registry numbers of firms for validation was not consistently available in the database on state aid.

\begin{figure}
    \centering
    \makebox[\textwidth][c]{\includegraphics[width=1\columnwidth]{Figures/FirmSizes}}%
    
    \decoRule
    \caption[Firm size distribution in dataset]{Histogram shows the distribution of total assets as a proxy for the size of the firms in the dataset. The X-axis is cut off at 10 Mio. Euros.}
    \label{fig:FirmSizes}
\end{figure}


%----------------------------------------------------------------------------------------
%	SECTION 3
%----------------------------------------------------------------------------------------

\section{Insolvency information}

To further understand how effective the aid measures were in preventing insolvencies, data on insolvencies in Germany was obtained from the insolvencies notification platform and matched with the beneficiaries of government support on the state aid transparency database. 
The data from the insolvencies notification platform data was used for the matching ranges from 2020 to March 2023. Like with the matching of financial information with the beneficiaries, matching was also prone to mismatches in the case of two companies of the same name.


% Chapter Template

\chapter{Methods} % Main chapter title

\label{Chapter4} % Change X to a consecutive number; for referencing this chapter elsewhere, use \ref{ChapterX}


%----------------------------------------------------------------------------------------
%	SECTION 1
%----------------------------------------------------------------------------------------

\section{Balance Sheet Ratios}
\label{section:BSratios}

To evaluate the financial position and performance of firms in a comparable way across the data set a selection of balance sheet ratios were chosen. Ratios allow a consistent view of the companies despite their different sizes. Even though balance sheets only offer a reporting date view of the firm's financial information and can't reflect events or extreme situations during a fiscal year, they provide a comparable view of companies that is standardized by accounting standards.  The selection of ratios was made to get a picture of the liquidity and solvency the of firms. The ratios are calculated for each beneficiary of government support for each available year between 2018 and 2021. Calculations are shown in Table \ref{tab:RatioCalc}.

\subsection{Liquidity Ratios}

Liquidity ratios are chosen to measure a firm's financial position to meet its obligations in the short run. As outlined in Chapter \ref{Chapter1}, the pandemic shock had a significant effect on companies' liquidity and was a key consideration for the EU to loosen state aid regulation and enable large scale support measures \parencite{eu_com_temporary_2020}. The first and most representative liquidity ratio is the cash ratio, comparing the most liquid asset, cash holdings, to the total assets of a firm. Cash is the starting buffer against running costs in a crisis shock. Although usually the current liabilities are used instead of the total assets, with the available data total assets serve as a more robust denominator that has been utilized in similar research \parencite{fernandez-cerezo_firm-level_2021, costa_state-aids_2021,igan_shot_2023}. The quick and the current ratio provide a more conservative view of a firm's liquidity by including assets that are still considered relatively liquid against the current (short-term) liabilities. However, the key component, is short term debt, is not disclosed consistently in balance sheets since accounting standards allow alternative discloser in the balance sheet appendix. For practicability, such a case the calculation had to use the total liabilities, which reduced the informative value in comparisons across firms, but still allows for comparisons on a firm level between different years.

\subsection{Solvency Ratios}

The other factor of interest is the indebtedness of the firm in the context of the pandemic and the remedial measures. The indebtedness, or also leverage, of a firm has implications that are rather relevant in the long-term, since debt payments are long term obligations that need to be serviced by cash flows. High levels of debt can challenge a company and can reduce profits. 
The debt-to-asset and the equity ratio compare the respective capital to the total assets and are behaving in opposite directions. The debt-to-equity ratio gives a magnified picture of the companies leverage compared to the debt-to-asset ratio. For the simplification purposes, negative ratios were omitted since result either from errors in the data parsing process or from exceptional cases like loss transfer agreements with parent companies. 

\begin{table}%[]
    \caption{The calculation of Balance Sheet Ratios.}
    \label{tab:RatioCalc}
    \centering
    \def\arraystretch{1.5}
    \begin{tabular}{@{}lll@{}}
    \toprule
    Category                   & Ratio                & Calculation \\ \midrule
    \multirow{3}{*}{Liquidity} & Cash Ratio           & $\frac{Cash}{Total Assets}$ \\ %\cmidrule(l){2-3} 
                                & Quick Ratio          & $\frac{Current Assets-Inventory}{Current Liabilities}$ \\ %\cmidrule(l){2-3} 
                                & Current Ratio        & $\frac{Current Assets}{Current Liabilities}$ \\ \midrule
    \multirow{3}{*}{Liability} & Debt-to-Equity Ratio & $\frac{Debt}{Equity}$ \\ %\cmidrule(l){2-3} 
                                & Equity Ratio         & $\frac{Equity}{Total Assets}$ \\ %\cmidrule(l){2-3} 
                                & Debt-to-Assets Ratio & $\frac{Debt}{Total Assets}$ \\ \bottomrule
    \end{tabular}
\end{table}








%----------------------------------------------------------------------------------------
%	SECTION 2
%----------------------------------------------------------------------------------------

\section{Difference-in-Differences}

With the obtained firm-level data the first analysis tries to (1) measure the causal treatment effect of government support during the COVID-19 pandemic and (2) explore how the effects between aid instruments on the ratios from section \ref{section:BSratios} differ. To estimate the causal effect of aid, the fact that aid measures were granted consecutively over the years 2020-2023 is used for a natural experiment with a standard difference-in-differences method to estimate the average treatment effect on the treated ($ATT$). In this setting the $ATT$ can be described as the average causal effect of an aid instrument on the balance sheet ratio of companies that received support. In mathematical terms can be described as follows:

\begin{equation}
    ATT = E[Y_{ratio,1i}| aid=1]-E[Y_{ratio,0i}| aid=1] 
    \label{eqn:ATT}
\end{equation}

The first term describes the expected balance sheet ratio amongst the companies that got aid. The second part describes the unobservable expected ratios of the very same group of companies if they won't have received any aid. In the quasi-experimental setup, the periods 2019 and 2020 will be compared, and companies that received support in 2020 serve as treated group, while companies that did not receive support in 2020, but later in 2021 or 2022 serve as control group. The classification in treated and untreated is based on the cut-off dates of the firm's balance sheets and the date of granting the support. The setting is also performed for the periods 2020 and 2021, where the companies in the control group only received aid in 2022. For the estimation of the difference-in-difference a regression with the following linear model is used:


\begin{equation}
    ratio_{ft} = \beta_{0} + \beta_{1}aid_{f} + \beta_{2}post_{t} + \beta_{3}aid\ast post_{ft} + \varepsilon_{ft} 
    \label{eqn:Diff&Diff}
\end{equation}

Where the dependent variable is the ratio for firm ($f$) in period ($t$). The coefficients on the right side of the equation are the dummy variables aid, post, their interaction term $ aid\ast post$ and the unobserved "error" term $ \varepsilon$. The first independent variable $aid$ equals 0 when firm ($f$) did not receive support until after the experiment period and is in the control group. The variable $aid$ equals 1 when the firm did receive support and is considered treated in the experiment period. The second independent variable $post$ indicates pre- and post-shock periods resembled by 0 and 1. The third and main variable for the difference-in-difference method is the interaction term $ aid\ast post$ which will only be 1 for a treated firm ($f=1$) in the treatment period ($t=1$). The coefficient of the interaction describes the change in the dependent variable due to the treatment as illustrated in Table \ref{tab:RatioCalc}. It can be seen that by inserting the values of aid and post the coefficient $\beta_{3}$ of the interaction of $aid$ and $post$ is the estimated difference-in-difference.

\begin{table}%[]
\caption{Difference-in-difference with regression}
\label{tab:DiDcoefficient}
\centering
\def\arraystretch{1.5}

\begin{tabular}{l|l|l|l|}
    \cline{2-4}
                                            & After($post=1$) & Before ($post=0$)& After - Before \\ \hline
    \multicolumn{1}{|l|}{Treated ($aid=1$)}           &  $\beta_{0}+\beta_{1}+\beta_{2}+\beta_{3}$     &  $\beta_{0}+\beta_{1}$      &   $\beta_{2}+\beta_{3}$             \\ \hline
    \multicolumn{1}{|l|}{Control ($aid=0$)}           &  $\beta_{0}+\beta_{2}$     &  $\beta_{0}$      &   $\beta_{2}$             \\ \hline
    \multicolumn{1}{|l|}{Treated - Control} &  $\beta_{1}+\beta_{3}$     &  $\beta_{1}$      &   $\beta_{3}$             \\ \hline
    \end{tabular}

\end{table}

The central assumption for the difference-in-difference methodology is that the control and the treatment group are comparable and would behave parallel in over the observed periods, if there would not be a treatment. The proposed setup is based on the assumption that companies who received aid at any point during the COVID-19 pandemic were sufficiently affected by the shock to be eligible for support and subsequently confirm the parallel trends assumption.

The methodology will be used to measure the effects of grants and loans separately for all ratios in 2020 as well as 2021. Due to the outliers in the ratios the upper and lower fifth percentile is omitted. 







%----------------------------------------------------------------------------------------
%	SECTION 3
%----------------------------------------------------------------------------------------

\section{Causal Curve}

Next, the effect of support during the COVID-19 pandemic will be further analyzed by considering the amount of support as a continuous variable, instead of a binary variable like in the difference-in-differences approach. Estimating the effect of a continuous treatment can be evaluated through generalized propensity scores and described by a dose-response function showing the causal effect at each dose of treatment. 

The groundwork was done by \parencite{hirano_propensity_2004}, by exploring the propensity scores methods for continuous treatments. 

Generally, propensity scores refer to the probability of a unit being exposed to a treatment (binary or continuous) given pre-exposure covariates \parencite{rosenbaum_central_1983}. 

The initial approach by Hirano and Imbens estimates the generalized propensity score (GPS) as the probability density function of the treatment conditioned on pre-exposure covariates, before combining the GPS with the treatment in a regression model to estimate a response to a treatment dose \parencite{hirano_propensity_2004}.

Compared to a traditional regression method, the GPS approach allows adjustment for possible confounders and the removal of bias in the estimation of the treatment effect, which can be crucial in observational studies \parencite{wu_matching_2021,moodie_estimation_2012}. 

Efforts by \parencite{moodie_estimation_2012} and \parencite{galagate_causal_2016} expanded on the concept of GPS' from Hirano and Imbens and proposed alternative estimators for dose-response functions.



This thesis uses the concept of transforming continuous treatments into probability density functions to approximate a dose-response for the government support. The estimation for both the generalized propensity scores and the dose-response function follows an implementation from \parencite{kobrosly_causal-curve_2020}. 
The approach consists of three steps. First, the probability of the treatment T, referring to the amount of aid, is modeled by conditioning it on a set of covariates X using a normal generalized linear model (GLM):


\begin{equation}
    T_{i}|X_{I}\sim \mathcal{N} (X^{T}_{i}\beta,\sigma^{2})
    \label{eqn:densit}
\end{equation}

The probability density function of the fitted values and the actual treatment is then estimated to get a GPS for each firm. To prevent possible confounding the variable shown in \ref{tab:Covariates} were used as covariates. 
In the second step, the outcome Y, representing the observed change in liquidity for each firm, is fitted on the observed treatment T and the GPS from the first step using a linear generalized additive model. 

\begin{table}%[]
    \caption{Covariates for GPS and dose-response function}
    \label{tab:Covariates}
    \centering
    \def\arraystretch{1.2}
    \begin{tabular}{ll}
        \hline
        Covariates                 & Explanation                                                                                                                                     \\ \hline
        aid\_loan\_2020            & \begin{tabular}[c]{@{}l@{}}Values of loans that the company \\ received in 2020\end{tabular}                                                    \\
        aid\_loan\_2021            & \begin{tabular}[c]{@{}l@{}}Values of loans that the company \\ received in 2021\end{tabular}                                                    \\
        aid\_loan\_2020            & \begin{tabular}[c]{@{}l@{}}Values of grants that the company \\ received in 2020\end{tabular}                                                   \\
        assets\_2020               & \begin{tabular}[c]{@{}l@{}}Total assets of the company in 2020 \\ as a proxy for company size\end{tabular}                                      \\
        debt\_to\_asset ratio 2020 & \begin{tabular}[c]{@{}l@{}}Ratio to reflect the indebtedness of \\ a company prior the aid is granted\end{tabular}                              \\
        days\_grant\_2021          & \begin{tabular}[c]{@{}l@{}}A variable that indicates the days between \\ the granting and the cutoff date of the \\ financial report\end{tabular} \\ \hline
        \end{tabular}
        \end{table}

    In the last step the model is used to predict the treatment response in liquidity at a multiple levels to of the aid payments to visualize the causal effect as a causal dose response curve (CDRC).

    The needed assumptions for the methods are that government support for one company will not affect the outcome for another company (Stable Unit Treatment Value Assumption). The only scenario where this could be violated is when two affiliated companies in the data set both received support but pool the money in one company. Second, the positivity assumption can also be confirmed since every company has some possibility of receiving support and the support is positive. The third assumption of unconfoundedness is most critical for observational studies. It has to be assumed that any relationship between potential outcomes in liquidity and the received support can be fully explained by the included covariates. By including the in Table \ref{tab:Covariates} presented covariates the issue of potential unobserved confounders was attempted to reduce best possible with the available data. How, it is likely that there are additional unmeasured confounders that weren't observed introduced bias to the results. 
        
% Chapter Template

\chapter{Results} % Main chapter title

\label{Chapter5} % Change X to a consecutive number; for referencing this chapter elsewhere, use \ref{ChapterX}

%----------------------------------------------------------------------------------------
%	SECTION 1
%----------------------------------------------------------------------------------------

\section{Balance Sheet Ratios}

The average observed liquidity ratios shown in Figure \ref{fig:Ratios} for all companies of the dataset are showing an increase in liquidity in 2020 and 2021 compared to the pre pandemic years indication that companies are holding relatively more cash at the year-end since the pandemic. A study conducted by the German Federal Bank reported an increase in the average cash ratio for German companies in 2020 as well as in 2021 \parencite{deutsche_bundesbank_jahresabschlussstatistik_2022}. For example, for SME corporations the study reported a change in the cash ratio from 0.104 (2019) to 0.110 (2020). For the current ratio and quick ratio, the same trend was reported. Further support for an increase in the quick ratio was found by another study \parencite{bley_mittelstand_2022}. Although the exact ratios are varying between studies, there is strong support for the general trend of increasing liquidity in 2020 and 2021.

Solvency ratios are showing a less clear trend after the COVID-19 pandemic. Although minimal, the opposite trends in the equity ratio and debt-to-asset-ratio are as expected. The only visible change happened in 2020, while in 2021 the ratios are very similar to 2018 and 2019. The change in the debt-to-asset ratio is amplified in the debt-to-equity ratio, as expected. Survy Data from the KFW found an Equity Ratio of 0.318 in 2019, a decrease to 0.301 in 2020 and a recovery to 0.314 in 2021 \parencite{kfw_kfw-mittelstandspanel_2022}. For very small companies with less than 10 employees, the drop in 2020 was stronger, and the recovery in 2021 was above pre-pandemic levels. Lager companies did not have a recovery after the crisis year and decreased their Equity Ratio in 2021 on average further. This could indicate that the recovery of the indebtedness in 2021 might have been driven by smaller companies. Similar observations were reported by the German Federal Bank were the debt-to-asset-ratio for SME corporations decreased in 2020.


\begin{figure}
\centering
\makebox[\textwidth][c]{\includegraphics[width=1.2\columnwidth]{Figures/chart_ratios}}%

\decoRule
\caption[Balance sheet ratios]{Boxplot with balance sheet ratios from the obained dataset.}
\label{fig:Ratios}
\end{figure}


%----------------------------------------------------------------------------------------
%	SECTION 2
%----------------------------------------------------------------------------------------

\section{Diff and Diff}

\begin{table}
    \caption{Government aid impact on ratios}
    \label{tab:DiDresults}
    \centering
    \def\arraystretch{1.8}
    \centering
    \begin{tabular}{llrr}
\toprule
                     & \textbf{year} &                2020 &                2021 \\
{} & \textbf{ratio} &                     &                     \\
\midrule
\textbf{cash ratio} & \textbf{grant} &    -0.006   (0.521) &    0.093*** (0.000) \\
                     & \textbf{loan} &    0.065*** (0.000) &   0.0388*** (0.000) \\
\textbf{quick ratio} & \textbf{grant} &   -0.0871   (0.457) &    0.0357   (0.853) \\
                     & \textbf{loan} &   0.0905**  (0.029) &      0.2039 (0.094) \\
\textbf{current ratio} & \textbf{grant} &   -0.1636   (0.204) &   -0.0279   (0.895) \\
                     & \textbf{loan} &   0.1518*** (0.001) &      0.2657 (0.072) \\
\textbf{debt to equity ratio} & \textbf{grant} &   0.5661**  (0.041) &   -0.3862   (0.126) \\
                     & \textbf{loan} &   1.0924*** (0.000) &      0.7023 (0.070) \\
\textbf{equity ratio} & \textbf{grant} &   -0.0095   (0.497) &    0.0136   (0.405) \\
                     & \textbf{loan} &  -0.0575*** (0.000) &  -0.0453*** (0.002) \\
\textbf{debt to assest ratio} & \textbf{grant} &   0.0366**  (0.024) &     -0.0319 (0.083) \\
                     & \textbf{loan} &   0.0764*** (0.000) &   0.0558*** (0.002) \\
\bottomrule
\end{tabular}
}

    \small  Notes: Standard errors in parentheses, *** p<0.01, ** p<0.05, * p<0.1

    \end{table}
    
%----------------------------------------------------------------------------------------
%	SECTION 3
%----------------------------------------------------------------------------------------

\section{Causal Curve}


% Chapter Template

\chapter{Conclusion} % Main chapter title

\label{Chapter6} % Change X to a consecutive number; for referencing this chapter elsewhere, use \ref{ChapterX}

%----------------------------------------------------------------------------------------
%	SECTION 1
%----------------------------------------------------------------------------------------

\section{Policy Implications}

To mitigate the effects of the pandemic shock the German government responded with unprecedented fiscal efforts to provide liquidity to the economy and prevent business from failing. Policy responses of this magnitude are complex in many ways. The regulative aspect regarding European state aid regulations has been covered shortly in the introduction.

Finding the right balance between the effectiveness and efficiency of the policy response is a key challenge. On the one hand the support needs to be appropriate to provide sufficient relief since an economic collapse is not a considerable alternative. On the other hand, support needs also be targeted, helping with liquidity where needed and not creating excessive benefits. 
Under the pressure to keep the economy creating targeted aid measures that adequate aid for everyone is hardly obtainable.

For example, the right level of assistance is very crucial for highly vulnerable companies that were already struggling before the pandemic shock. Legitimate concerns about the side effects of generous support to this group of companies are present. The zombification, which refers to artificially keeping these companies alive and delaying inevitable default, as well as the distortion of competition are part of the concerns \parencite{dorr_small_2022}. But also, considerations of fairness and the interpretation to whom the fairness should be interpreted. Policy makers were faced with difficult tradeoffs, especially given the possible chaining on insolvencies, which could have unmanageable consequences and possibly counteract the actual goals of the aid and hamper the effort that was already made.

A method that was utilized through grants schemes were provisional permits with expedited payouts for quick relief, but subject to repayment. Thus, allowing for a final determination of the granted amount at a later date. 

In regard to the assessment of support measures no judgment on observed excessive liquidity is possible, since funds could still be due for repayment.




%----------------------------------------------------------------------------------------
%	SECTION 2
%----------------------------------------------------------------------------------------

\section{Conclusion}

With the DiD regression effects from aid on the liquidity and solvency of companies were successfully measured. The results suggest that in 2020 loan-based measures supported companies by improving their liquidity by 5.3 \%, while grant-based aid was insufficient as a liquidity injection since beneficiaries had no improvement in liquidity and relatively higher debt. However, in 2021 grants programs became effective in providing liquidity and preserving the equity of firms. In 2021 the liquidity increase of grant beneficiaries was 7.7 \%, twice as high as the effect of grants in 2021. 

Moreover, the insolvencies of aid beneficiaries were analyzed. Results suggest that they were already significantly weaker before the pandemic and that aid measures were less effective in supporting them compared to the overall population of beneficiaries. However, a sector view reveals that the industries that are most represented amongst beneficiaries gastronomy and accommodation show insolvency rates well below the average, suggesting that aid measures were successful in reducing the chance of insolvency in their presumed main target sectors.

Finally, the thesis provided a deeper understanding of the aid measures by exploiting the data granularity with a combination of generalized propensity scores and a generalized linear model (GLM). The visualizations were showing the relationship between changes in beneficiaries' liquidity as well as solvency at different aid levels. Overall, the visualizations are in line with the results from the DiD and don't show any concerning abnormalities. In greater detail, the visualizations reveal some heterogeneity amongst beneficiaries from different industries, which connects with the observed heterogeneity in insolvencies of different industries.

With the results obtained, limitations must also be recognized. In addition to limitations arising from the data sources and the inherent risk of mismatches during that matching process, the causal modeling for the DiD and the GPS involved strong assumptions such as the assumption of parallel trends and unconfoundedness that, if violated, can heavily bias the estimates.
 
In consideration of the overall results of the thesis and already existing contributions from other researchers the government support had a considerable role in carrying the economy through the pandemic. However, the future will require more in-depth assessments that fully reflect the repayments of excessive liquidity support and have a longer-term view on beneficiaries' insolvencies.


%\include{Chapters/Chapter1}
%\include{Chapters/Chapter2} 
%\include{Chapters/Chapter3}
%\include{Chapters/Chapter4} 
%\include{Chapters/Chapter5} 

%----------------------------------------------------------------------------------------
%	THESIS CONTENT - APPENDICES
%----------------------------------------------------------------------------------------

\appendix % Cue to tell LaTeX that the following "chapters" are Appendices

% Include the appendices of the thesis as separate files from the Appendices folder
% Uncomment the lines as you write the Appendices

% Appendix A

\chapter{Frequently Asked Questions} % Main appendix title

\label{AppendixA} % For referencing this appendix elsewhere, use \ref{AppendixA}

\section{How do I change the colors of links?}

The color of links can be changed to your liking using:

{\small\verb!\hypersetup{urlcolor=red}!}, or

{\small\verb!\hypersetup{citecolor=green}!}, or

{\small\verb!\hypersetup{allcolor=blue}!}.

\noindent If you want to completely hide the links, you can use:

{\small\verb!\hypersetup{allcolors=.}!}, or even better: 

{\small\verb!\hypersetup{hidelinks}!}.

\noindent If you want to have obvious links in the PDF but not the printed text, use:

{\small\verb!\hypersetup{colorlinks=false}!}.


\begin{table}
    \caption{Complete coefficients for government aid impact on ratios}
    \label{tab:DiDresultsAll}
    \centering
    \def\arraystretch{1.2}
    \centering
    \begin{tabular}{llllll}
\toprule
     &      &                      &                aid &            post &           aid*post \\
type & year & ratio &                    &                 &                    \\
\midrule
grant & 2020 & cash ratio &    -0.0069 (0.551) &  -0.005 (0.300) &   0.0366** (0.024) \\
     &      & quick ratio &    -0.0069 (0.551) &  -0.005 (0.300) &   0.0366** (0.024) \\
     &      & current ratio &    -0.0069 (0.551) &  -0.005 (0.300) &   0.0366** (0.024) \\
     &      & debt to equity ratio &    -0.0069 (0.551) &  -0.005 (0.300) &   0.0366** (0.024) \\
     &      & equity ratio &    -0.0069 (0.551) &  -0.005 (0.300) &   0.0366** (0.024) \\
     &      & debt to assest ratio &    -0.0069 (0.551) &  -0.005 (0.300) &   0.0366** (0.024) \\
loan & 2020 & cash ratio &  0.0831*** (0.000) &  -0.005 (0.251) &  0.0764*** (0.000) \\
     &      & quick ratio &  0.0831*** (0.000) &  -0.005 (0.251) &  0.0764*** (0.000) \\
     &      & current ratio &  0.0831*** (0.000) &  -0.005 (0.251) &  0.0764*** (0.000) \\
     &      & debt to equity ratio &  0.0831*** (0.000) &  -0.005 (0.251) &  0.0764*** (0.000) \\
     &      & equity ratio &  0.0831*** (0.000) &  -0.005 (0.251) &  0.0764*** (0.000) \\
     &      & debt to assest ratio &  0.0831*** (0.000) &  -0.005 (0.251) &  0.0764*** (0.000) \\
grant & 2021 & cash ratio &   -0.029** (0.026) &  0.0164 (0.323) &   -0.0319* (0.083) \\
     &      & quick ratio &   -0.029** (0.026) &  0.0164 (0.323) &   -0.0319* (0.083) \\
     &      & current ratio &   -0.029** (0.026) &  0.0164 (0.323) &   -0.0319* (0.083) \\
     &      & debt to equity ratio &   -0.029** (0.026) &  0.0164 (0.323) &   -0.0319* (0.083) \\
     &      & equity ratio &   -0.029** (0.026) &  0.0164 (0.323) &   -0.0319* (0.083) \\
     &      & debt to assest ratio &   -0.029** (0.026) &  0.0164 (0.323) &   -0.0319* (0.083) \\
loan & 2021 & cash ratio &  0.0962*** (0.000) &  0.0151 (0.288) &  0.0558*** (0.002) \\
     &      & quick ratio &  0.0962*** (0.000) &  0.0151 (0.288) &  0.0558*** (0.002) \\
     &      & current ratio &  0.0962*** (0.000) &  0.0151 (0.288) &  0.0558*** (0.002) \\
     &      & debt to equity ratio &  0.0962*** (0.000) &  0.0151 (0.288) &  0.0558*** (0.002) \\
     &      & equity ratio &  0.0962*** (0.000) &  0.0151 (0.288) &  0.0558*** (0.002) \\
     &      & debt to assest ratio &  0.0962*** (0.000) &  0.0151 (0.288) &  0.0558*** (0.002) \\
\bottomrule
\end{tabular}
}

    \small  Notes: Standard errors in parentheses, *** p<0.01, ** p<0.05, * p<0.1

    \end{table}


% Appendix B

\chapter{Full list of insolvent aid  beneficiaries by industry} % Main appendix title

\label{AppendixB} % 

    \begin{table}
        \caption{Share of insolvent aid  beneficiaries by industry}
        \label{tab:InsByIndustry}
        \centering
        \def\arraystretch{1}
        \centering
        \begin{tabular}{lrrl}
\toprule
                                industry &  aid beneficiaries &  insolvent & share \\
\midrule
                   Employment activities &                594 &         30 & 5.05\% \\
               Construction of buildings &               1247 &         37 & 2.97\% \\
             Manufacture of basic metals &                586 &         14 & 2.39\% \\
Computer programming, consultancy and re &               1481 &         35 & 2.36\% \\
Manufacture of machinery and equipment n &               2226 &         44 & 1.98\% \\
Manufacture of computer, electronic and  &                728 &         13 & 1.79\% \\
     Specialised construction activities &               4103 &         68 & 1.66\% \\
Manufacture of rubber and plastic produc &                664 &         10 & 1.51\% \\
Printing and reproduction of recorded me &                826 &         11 & 1.33\% \\
                     Other manufacturing &                689 &          9 & 1.31\% \\
Architectural and engineering activities &                929 &         12 & 1.29\% \\
Manufacture of fabricated metal products &               3105 &         40 & 1.29\% \\
Land transport and transport via pipelin &               2552 &         31 & 1.21\% \\
            Manufacture of food products &               1537 &         17 & 1.11\% \\
         Gambling and betting activities &               2061 &         22 & 1.07\% \\
Wholesale trade, except of motor vehicle &               5396 &         56 & 1.04\% \\
       Other personal service activities &               2517 &         25 & 0.99\% \\
Services to buildings and landscape acti &               1018 &         10 & 0.98\% \\
Retail trade, except of motor vehicles a &               8810 &         84 & 0.95\% \\
Warehousing and support activities for t &               1620 &         15 & 0.93\% \\
Office administrative, office support an &               3040 &         28 & 0.92\% \\
         Advertising and market research &                952 &          8 & 0.84\% \\
                  Real estate activities &               1413 &         11 & 0.78\% \\
                Manufacture of furniture &                530 &          4 & 0.75\% \\
Activities of head offices; management c &                967 &          6 & 0.62\% \\
    Food and beverage service activities &              15173 &         88 & 0.58\% \\
           Rental and leasing activities &                868 &          5 & 0.58\% \\
Sports activities and amusement and recr &               5232 &         27 & 0.52\% \\
                               Education &                974 &          5 & 0.51\% \\
Wholesale and retail trade and repair of &               3525 &         17 & 0.48\% \\
Creative, arts and entertainment activit &               1485 &          6 & 0.40\% \\
                 Human health activities &               1417 &          5 & 0.35\% \\
Travel agency, tour operator and other r &               2756 &          8 & 0.29\% \\
Motion picture, video and television pro &                691 &          2 & 0.29\% \\
                           Accommodation &               9885 &         24 & 0.24\% \\
Crop and animal production, hunting and  &               1803 &          4 & 0.22\% \\
         Legal and accounting activities &                646 &          1 & 0.15\% \\
\bottomrule
\end{tabular}
}
    
        \small  Notes: Tables shows industries with more than 500 beneficiaries
    
        \end{table}
%% Appendix B
\chapter{Appendix B}
\label{AppendixB} % 
%\chapter{Full list of insolvent aid  beneficiaries by industry} % Main appendix title
\section{Full list of insolvent aid  beneficiaries by industry}


\begin{table}
        \caption{Share of insolvent aid  beneficiaries by industry}
        \label{tab:InsByIndustry}
        \centering
        \def\arraystretch{1}
        \centering
        \begin{tabular}{lrrl}
\toprule
                                industry &  aid beneficiaries &  insolvent & share \\
\midrule
                   Employment activities &                594 &         30 & 5.05\% \\
               Construction of buildings &               1247 &         37 & 2.97\% \\
             Manufacture of basic metals &                586 &         14 & 2.39\% \\
Computer programming, consultancy and re &               1481 &         35 & 2.36\% \\
Manufacture of machinery and equipment n &               2226 &         44 & 1.98\% \\
Manufacture of computer, electronic and  &                728 &         13 & 1.79\% \\
     Specialised construction activities &               4103 &         68 & 1.66\% \\
Manufacture of rubber and plastic produc &                664 &         10 & 1.51\% \\
Printing and reproduction of recorded me &                826 &         11 & 1.33\% \\
                     Other manufacturing &                689 &          9 & 1.31\% \\
Architectural and engineering activities &                929 &         12 & 1.29\% \\
Manufacture of fabricated metal products &               3105 &         40 & 1.29\% \\
Land transport and transport via pipelin &               2552 &         31 & 1.21\% \\
            Manufacture of food products &               1537 &         17 & 1.11\% \\
         Gambling and betting activities &               2061 &         22 & 1.07\% \\
Wholesale trade, except of motor vehicle &               5396 &         56 & 1.04\% \\
       Other personal service activities &               2517 &         25 & 0.99\% \\
Services to buildings and landscape acti &               1018 &         10 & 0.98\% \\
Retail trade, except of motor vehicles a &               8810 &         84 & 0.95\% \\
Warehousing and support activities for t &               1620 &         15 & 0.93\% \\
Office administrative, office support an &               3040 &         28 & 0.92\% \\
         Advertising and market research &                952 &          8 & 0.84\% \\
                  Real estate activities &               1413 &         11 & 0.78\% \\
                Manufacture of furniture &                530 &          4 & 0.75\% \\
Activities of head offices; management c &                967 &          6 & 0.62\% \\
    Food and beverage service activities &              15173 &         88 & 0.58\% \\
           Rental and leasing activities &                868 &          5 & 0.58\% \\
Sports activities and amusement and recr &               5232 &         27 & 0.52\% \\
                               Education &                974 &          5 & 0.51\% \\
Wholesale and retail trade and repair of &               3525 &         17 & 0.48\% \\
Creative, arts and entertainment activit &               1485 &          6 & 0.40\% \\
                 Human health activities &               1417 &          5 & 0.35\% \\
Travel agency, tour operator and other r &               2756 &          8 & 0.29\% \\
Motion picture, video and television pro &                691 &          2 & 0.29\% \\
                           Accommodation &               9885 &         24 & 0.24\% \\
Crop and animal production, hunting and  &               1803 &          4 & 0.22\% \\
         Legal and accounting activities &                646 &          1 & 0.15\% \\
\bottomrule
\end{tabular}
}
    
        \small  Notes: Tables shows industries with more than 500 beneficiaries
    
\end{table}

%
% Appendix C
\chapter{Appendix C}
%\chapter{Additional curves for selected industry } % Main appendix title


%\label{AppendixC} % 



% 1
\section{Figure 5.3 with full Axis range}

\begin{figure}
    \centering
    \makebox[\textwidth][c]{\includegraphics[width=1\columnwidth]{Figures/causal_curves1_raw.png}}
    
    \decoRule
    \caption[Response curves for grants and loans uncut]{Estimated Dose Response Functions, for liquidity (cash) from grants 2021 (top) and loans 2020 (bottom) in relative (left) and absolte (right) terms with 95\% Confidence Bands. For Binomial Distributed Data. The estimate of absolute variants used a lognormal GLM for the GPS estimation due to the different distribution of the variables.}
    \label{fig:Curve1raw}
\end{figure}


% 2
\section{Figure 5.4 with full Axis range}

\begin{figure}
    \centering
    \makebox[\textwidth][c]{\includegraphics[width=1\columnwidth]{Figures/causal_curves2_raw.png}}
    
    \decoRule
    \caption[Response curves for indebtedness through aid uncut]{Estimated Dose Response Functions, for the debt-to-asset ratio from grants 2021 and loans 2021 in relative terms with 95\% Confidence Bands}
    \label{fig:Curve2raw}
\end{figure}



% 3
\section{Figure 5.5 with full Axis range}

\begin{figure}
    \centering
    \makebox[\textwidth][c]{\includegraphics[width=1\columnwidth]{Figures/causal_curves_industries1_RAW.png}}
    
    \decoRule
    \caption[Response curves for liquidity through aid - by sectors 1 uncut]{Estimated Dose Response Functions, for the cash ratio from grants 2021 in selected industries in relative terms with 95\% Confidence Bands}
    \label{fig:Curve3raw}
\end{figure}



% 4
\section{Dose curves from addtional industries}

\begin{figure}
    \centering
    \makebox[\textwidth][c]{\includegraphics[width=1\columnwidth]{Figures/causal_curves_industries2.png}}
    
    \decoRule
    \caption[Response curves for liquidity through aid - by sectors 2]{Estimated Dose Response Functions, for the cash ratio from grants 2021 in selected industries in relative terms with 95\% Confidence Bands}
    \label{fig:Curve5}
\end{figure}

\begin{figure}
    \centering
    \makebox[\textwidth][c]{\includegraphics[width=1\columnwidth]{Figures/causal_curves_industries2_raw.png}}
    
    \decoRule
    \caption[Response curves for liquidity through aid - by sectors 2 uncut]{Estimated Dose Response Functions, for the cash ratio from grants 2021 in selected industries in relative terms with 95\% Confidence Bands}
    \label{fig:Curve5raw}
\end{figure}

%----------------------------------------------------------------------------------------
%	BIBLIOGRAPHY
%----------------------------------------------------------------------------------------

\printbibliography[heading=bibintoc]

%----------------------------------------------------------------------------------------
\def\authorshipname{Statement of Authorship}
\begin{declaration}
	\addchaptertocentry{\authorshipname} % Add the declaration to the table of contents
	\noindent I hereby confirm and certify that this master thesis is my own work. 
	All ideas and language of others are acknowledged in the text. 
	All references and verbatim extracts are properly quoted 
	and all other sources of information are specifically and clearly designated. 
	I confirm that the digital copy of the master thesis that 
	I submitted on 02.05.2023 is identical to the printed version 
	I submitted to the Examination Office on 03.05.2023.\\
	 
	\noindent DATE:\\
	%\rule[0.5em]{25em}{0.5pt} % This prints a line for the signature
	 
	\noindent NAME:\\
	%\rule[0.5em]{25em}{0.5pt} % This prints a line to write the date

	\noindent SIGNATURE:\\
	%\rule[0.5em]{25em}{0.5pt} % This prints a line to write the date
	\end{declaration}


\end{document}  
