% Chapter Template

\chapter{Introduction to the government support in Germany} % Main chapter title

\label{Chapter1} % Change X to a consecutive number; for referencing this chapter elsewhere, use \ref{ChapterX}


%----------------------------------------------------------------------------------------
%	No SubsECTION 
%----------------------------------------------------------------------------------------


The Covid-19 pandemic has severely affected the entire world with many devastating consequences. Businesses in many parts of the economy were struggling to survive due to shocks in demand, lockdowns from governments and disrupted supply chains \parencite{eu_com_temporary_2020}.

To sustain the economy and prevent businesses them from bankruptcy during the pandemic, the German government responded with a range of policies. Beside the various measures like labor cost subsidies, temporary changes in the insolvency law and tax reliefs, the financial support through grants and loans was unprecedented. The financial support was available for businesses in all sizes that affected by the pandemic ranging from self-employed individuals to small and medium-sized enterprises (SMEs) up to very large companies. From spring 2020 to summer 2022, grants, loans, recapitalizations and guarantees alone accounted for a total of around EUR 130 billion \parencite{bmwk_uberblickspapier_2022}. 

Grants and loans were the two groups of instruments with widespread use by the German government. Grants are funds provided by the government to businesses that are not needed to be repaid. Grants are usually subject to the terms and conditions, but do not require any consideration in return. Whereas financial support measures based on loans have to be repaid, like standard bank loans. The advantage over a normal credit transaction are beneficial conditions that a company would not have received under normal circumstances from a bank. Especially not in times where the company's future is uncertain and linked to the further development of the pandemic. 

From the companies' point of view, both types of aid have the immediate effect of a liquidity injection, meaning that additional cash is available. However, despite the liquidity additional debt doesn't offset deteriorating effect of potential losses on the equity of firms. Together with the repayment obligation, loans have a contrasting effect on the solvency compared to grants which could only become relevant in the long run.

Irrespective of the concrete form of the aid measures their scale manifests a fiscal effort that is inconceivable at this magnitude under normal conditions. Usually, governments are not permitted to provide extensive subsidies, due to concerns of distorting competing in the European single market \parencite{claici_european_2022}. The permissibility of subsidies is comprehensively regulated by European state aid laws. Before a subsidy is considered permissible under this legal framework an assessment of its necessity, incentive effect, proportionality and effect on trade and competition is needed \parencite{claici_european_2022}. In light of the ongoing pandemic, the EU relaxed rules on subsidies by introducing the Temporary Framework for State aid measures to support the economy in the current COVID-19 outbreak, which provided national governments more freedom to come up with quick and extensive policy responses to support businesses \parencite{eu_com_temporary_2020}.

Under the framework a justification the financial support measures including the necessity, the appropriateness and proportionality to remedy the impact of the pandemic in the economy was required from the German government \parencite{eu_com_temporary_2020}. Defining and deciding on the appropriateness as well as proportionality of support measures is a complex and challenging task. Due to the unpredictable scale of the pandemic, uncertainty is immense. On the other hand, the effect of support measures is nothing trivial to estimate, given that their scale was unprecedented. To ensure that the support measures are effective, efficient, a good understanding of aid instruments is vital. 

A key role of assessments is to provide this understanding. This thesis aims to access whether grants and loans were successful in in providing the much-needed liquidity and preventing businesses from bankruptcy.

In specific, this thesis tries to answer whether the aid measures were successful in providing liquidity, by measuring the causal effect of aid instruments on the firm liquidity. A causal effect that is more than marginal would allow to attribute the liquidity increase to the aid program and bring support for their successfulness. In case of an unmeasurable or marginal effects the injected liquidity could have been fully absorbed by aid recipients or have been generally ineffective. Questions about whether the aid was providing sufficient support 
would arise and, in any case, challenge the successfulness of liquidity support through grants and loans. Considering the magnitude of government intervention and the scale of fiscal efforts aid measures are expected to a measurable effect on firm liquidity.

With regard to the aforementioned differences between grants and loans this thesis will also seek to answer whether these instruments have different effects on the solvency of aid beneficiaries. It is expected that loans increase the leverage, while grants are expected to have the opposite effect and help to preserve the equity of firms.

The analysis of liquidity and solvency will be supplemented by a view on dose response functions to provide a granular picture of the effects at different levels of aid. The insights will be used to answer whether the effects of aid are varying depending on the level of aid. 

And with a sectoral overview the assumption of heterogenous effects of grants between industries with different cost structures from \parencite{bischof_bedeutung_2021} will be tested.

Finally, this paper looks closer at aid beneficiaries that are insolvent by spring 2023 to compare the effectiveness of their aid as well as the heterogeneity assumption.

The approach differs from existing assessments in two regards. First, only publicly available data sources are used. Previous assessment of pandemic aid with a focus on Germany were only conducted on the basis of survey data \parencite{marek_impact_2022,bertschek_german_2022,dorr_small_2022,bischof_bedeutung_2021}. As such, this work demonstrates a way of policy evaluation that is not reliant on surveys, which are typically time and resource intensive.

Second, an approach with generalized propensity scores is adopted, which has so far only played a role on the context of other subsidy assessments \parencite{selebaj_effects_2021,carboni_effect_2017}. This methodology allows the estimation and visualization of dose-response functions to determine whether the causal effects support vary depending on the different size of the aid measures \parencite{selebaj_effects_2021}.

This thesis is organized as follows. After this introduction, chapter \ref{Chapter2} provides an overview of the general effects of COVID-19 pandemic on businesses as well as the already existing impact assessments of government support. Chapter \ref{Chapter3} presents the data sources and chapter \ref{Chapter4} explains the methodologies. In chapter \ref{Chapter5}, the findings about the effect of aid measures are shown. Chapter \ref{Chapter6} addresses the policy implications and concludes. 





\begin{table}
\caption{Overview of support instruments}
\label{tab:treatments}
\centering
\begin{tabular}{lr}
\toprule
                   Beihilfeinstrument &               aid \\
\midrule
Andere Formen der Kapitalintervention &  9,017,729,574.33 \\
                           Bürgschaft &  1,144,042,410.02 \\
              Eigenkapitalinstrumente &  2,419,881,701.00 \\
      Kredite/rückzahlbare Vorschüsse &    753,217,635.27 \\
            Sonstiges (bitte angeben) & 14,244,894,962.69 \\
               Zinsgünstiges Darlehen & 10,500,942,385.00 \\
                         Zinszuschuss &  9,383,307,910.00 \\
                             Zuschuss & 24,959,647,770.34 \\
\bottomrule
\end{tabular}
}

\end{table}
    

%\caption{The effects of treatments X and Y on the four groups studied.}

