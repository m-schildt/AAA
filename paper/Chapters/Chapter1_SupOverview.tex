% Chapter Template

\chapter{Introduction to the government support in Germany} % Main chapter title

\label{Chapter1} % Change X to a consecutive number; for referencing this chapter elsewhere, use \ref{ChapterX}


%----------------------------------------------------------------------------------------
%	No SubsECTION 
%----------------------------------------------------------------------------------------


The Covid-19 pandemic has severely affected the entire world with many devastating consequences. Businesses in many parts of the economy were struggling to survive due to shocks in demand, lockdowns from governments and disrupted supply chains \parencite{eu_com_temporary_2020}.

To sustain the economy and prevent businesses them from bankruptcy during the pandemic, the German government responded with a range of policies. Beside the various measures like labor cost subsidies, temporary changes in the insolvency law and tax reliefs, the financial support through grants and loans was unprecedented. The financial support was available for businesses in all sizes that affected by the pandemic ranging from self-employed individuals to small and medium-sized enterprises (SMEs) up to very large companies. From spring 2020 to summer 2022, grants, loans, recapitalizations and guarantees alone accounted for a total of around EUR 130 billion \parencite{bmwk_uberblickspapier_2022}. A fiscal effort of this magnitude is inconceivable under normal conditions.

Usually, governments are not permitted to provide extensive subsidies, due to concerns of distorting competing in the European single market \parencite{claici_european_2022}. 
The permissibility of subsidies is comprehensively regulated by European state aid laws. Before a subsidy is considered permissible under this legal framework an assessment of its necessity, incentive effect, proportionality and effect on trade and competition is needed \parencite{claici_european_2022}. 
In light of the ongoing pandemic, the EU relaxed rules on subsidies by introducing the Temporary Framework for State aid measures to support the economy in the current COVID-19 outbreak, by which provided national governments more freedom in order to come up with quick and extensive policy responses to support businesses \parencite{eu_com_temporary_2020}.

As part of the framework the German government had to justify the financial support measures by laying out the necessity, the appropriateness and proportionality to remedy the impact of the pandemic in the economy. Defining and deciding on the appropriateness as well as proportionality of support measures is a complex and challenging task. Due to the unpredictable scale of the pandemic, uncertainty is immense. On the other hand, the effect of support measures is nothing trivial to estimate, given that their scale was unprecedented. To ensure that the support measures are effective, efficient, a good understanding is inevitable.

The financial support measures introduced by the German government can mainly be categorized into the groups grants and loans. Grants are funds provided by the government to businesses that are not needed to be repaid. Grants are usually subject to the terms and conditions, but do not require any consideration in return. Whereas financial support measures based on loans have to be repaid, like standard bank loans. The advantage over a normal credit transaction are beneficial conditions that a company would not have received under normal circumstances from a bank. Especially not in a time where the company’s future is uncertain and linked to the further development of the pandemic.

From the companies' point of view, both types of aid have the immediate effect of a liquidity injection, meaning that additional cash is available. However, in the long-term perspective the repayment obligation of loans is contrasting the effect of grants by the fact that a firm will have to service the debt and interest of the loan, regardless of whether the pandemic is over or not.



\begin{table}
\caption{Overview of support instruments}
\label{tab:treatments}
\centering
\begin{tabular}{lr}
\toprule
                   Beihilfeinstrument &               aid \\
\midrule
Andere Formen der Kapitalintervention &  9,017,729,574.33 \\
                           Bürgschaft &  1,144,042,410.02 \\
              Eigenkapitalinstrumente &  2,419,881,701.00 \\
      Kredite/rückzahlbare Vorschüsse &    753,217,635.27 \\
            Sonstiges (bitte angeben) & 14,244,894,962.69 \\
               Zinsgünstiges Darlehen & 10,500,942,385.00 \\
                         Zinszuschuss &  9,383,307,910.00 \\
                             Zuschuss & 24,959,647,770.34 \\
\bottomrule
\end{tabular}
}

\end{table}
    

%\caption{The effects of treatments X and Y on the four groups studied.}

