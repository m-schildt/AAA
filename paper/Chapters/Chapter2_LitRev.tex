% Chapter Template

\chapter{Literature Review} % Main chapter title

\label{Chapter2} % Change X to a consecutive number; for referencing this chapter elsewhere, use \ref{ChapterX}

%----------------------------------------------------------------------------------------
%	SECTION 1
%----------------------------------------------------------------------------------------

\section{Pandemic effects}

The negative consequences of the COVID-19 pandemic on the economy have become evident in many areas. Many businesses were severely affected by drops in demand and through lockdowns ordered by authorities. When business operations are halting while costs like rent or personal costs continue occurring the pandemic shock eventually leads to negative cash flows for many firms \parencite{fernandez-cerezo_firm-level_2021}.
Depending on the affectedness of the business, the liquidity reserves will inevitably deteriorate and eventually liquidity shortfalls are inescapable with negative cash flows \parencite{puhr_have_2021}.
Although the demand of liquidity is individual for every company, a simulation study from Italy quantified the total liquidity deficit of all Italian SMEs caused by the Covid-19 shock to 83.7 billion Euros at the end of 2020 \parencite{bellucci_consequences_2022}. 
In comparison for the Belgian corporate sector, in a scenario without policy interventions, the drop in liquidity by September 2020 is quantified with 28.2 billion Euros \parencite{tielens_belgian_2020}. 

Empirical results from Tielens et al. (2020) suggest that in Belgium even businesses that used to be profitable require a large amount of additional financing to offset their liquidity shortfall.

Without a return of profits, firms are in need of liquidity injection to, either through additional equity or via debt. For smaller unlisted firms it is usually not possible to easily raising equity, therefore they usually left with the debt option and rely on credits from banks \parencite{pagano_covid-19_2022}. Pagano and Zechner \parencite*{pagano_covid-19_2022} analyzed the effects of covid 19 on companies’ financial performance in the EU. Their evidence suggests differences in the effects between large firms and small and medium sized enterprises. By comparing the years 2019 and 2020 the authors found that smaller companies tend to increase their ratio of total debt to total assets (debt-ratio) whereas, large companies also increase their leverage, but significantly less. Regarding liquidity, their findings suggest that small and medium sized enterprises increased their cash-to-total-assets-ratio more than large companies. Small companies did so even more than medium sized ones. However, the authors could only speculate over the reason behind of this observation. Plausible reasons were precautionary cash hording and greater risk aversion. Additionally, the authors raise the theory that smaller companies were able to raise cash more easily due to the claim, that loan guarantee programs favored small firms. However, the analyzed sample of small and medium sized enterprises was not representative of any specific industry, nor of aid recipients.

However, credits from banks, if obtainable, increases the firms leverage and could make the firm vulnerable to new liquidity shortfalls. And additional leverage only prevents from insolvency if there is a prospect that future cash flows will enable a firm to service the additional debt. Regarding the capital structure, increased leverage means a weaker equity ratio. An early Survey study from September 2020 analyzed the implications of the pandemic crisis on the equity of Germany companies and reported that for most companies the equity ratio did not change, however a strong sectoral heterogeneity with travel and gastronomy having a reduction in the equity ratio between 1.8 \% and 1.5 \% \parencite{peichl_eigenkapitalentwicklung_2021}. In Spain a survey looked at the indebtedness as well as the cash ratio of enterprises and reported findings that support the heterogeneity of the covid 19 shock across firms and, that the impact was larger for small, young and less productive firms located in urban areas \parencite{fernandez-cerezo_firm-level_2021}. Further support for the heterogeneity of the impact of the COVID-19 crisis on firms’ sales and costs came from Belgium \parencite{dhyne_belgian_2021}.

A simulation on 14 relatively well-covered European countries estimated that an increase in the financial debt of companies has on average a negative impact on the growth of investment after the crisis, indicating negative long-term effects increased leverage \parencite{demmou_insolvency_2021}.



%----------------------------------------------------------------------------------------
%	SECTION 2
%----------------------------------------------------------------------------------------

\section{Government support effects}

Policy intervention to "prevent” the effects and save businesses for a fast economic recovery.
First assessments were modeling approaches.
Already lots of early assessments of state aid, also at a firm level.
For getting a better understanding on the effect of aid schemes in Germany a paper analyses the effect of a company’s cost structure on the effectives of aid measures (Bischof, Karlsson, Rostam-Aschar, Simon, 2021). 
This paper assumes that companies within the same sector have a similar cost structure. 
Since aid in Germany is based on the cost structure of companies, the authors conclude that based on the generalized approach of aid schemes, the effectiveness of aid is varying between business sectors.