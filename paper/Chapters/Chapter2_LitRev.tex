% Chapter Template

\chapter{Literature Review} % Main chapter title

\label{Chapter2} % Change X to a consecutive number; for referencing this chapter elsewhere, use \ref{ChapterX}

%----------------------------------------------------------------------------------------
%	SECTION 1
%----------------------------------------------------------------------------------------

\section{Pandemic effects}

The negative consequences of the COVID-19 pandemic on the economy have become evident in many areas. Many businesses were severely affected by drops in demand and containment measures ordered by authorities \parencite{eu_com_temporary_2020}.

When business operations are halting while costs like rent or personal costs continue occurring the pandemic shock eventually leads to negative cash flows for many firms \parencite{fernandez-cerezo_firm-level_2021}.
Depending on the affectedness of the business, the liquidity reserves will inevitably deteriorate and eventually liquidity shortfalls are inescapable with negative cash flows \parencite{puhr_have_2021}. Although the demand for liquidity is individual for every company, the overall lack of liquidity is apparent. 

An early simulation study from Italy quantified the total liquidity deficit of all Italian SMEs caused by the Covid-19 shock to 83.7 billion Euros at the end of 2020 \parencite{bellucci_consequences_2022}. In comparison to the Belgian corporate sector, in a scenario without policy interventions, the drop in liquidity by September 2020 was quantified at 28.2 billion Euros \parencite{tielens_belgian_2020}. Empirical results from \parencite{tielens_belgian_2020} suggest that in Belgium even businesses that used to be profitable require a large amount of additional financing to offset their liquidity shortfall. 
However, evidence from Belgium also showed that the COVID-19 pandemic widened the gaps between companies with some performing than in normal times \parencite{dhyne_belgian_2021}.  


Without a return of profits, firms are in need of liquidity injection, either through additional equity or via debt. For smaller unlisted firms it is usually not feasible to raise the raise equity, therefore they are usually left with the debt option and rely on credits from banks \parencite{pagano_covid-19_2022}. Pagano and Zechner \parencite*{pagano_covid-19_2022} analyzed the effects of covid 19 on European companies’ financial performance and found evidence suggesting differences in the effects between large firms and small and medium sized enterprises. By comparing the years 2019 and 2020 the authors found that smaller companies tend to increase their ratio of total debt to total assets (debt-ratio) whereas, large companies also increase their leverage, but significantly less. Regarding liquidity, their findings suggest that small and medium sized enterprises increased their cash-to-total-assets-ratio more than large companies. Small companies did so even more than medium sized ones. However, the authors could only speculate on the reason behind this observation. Plausible reasons were precautionary cash hoarding and greater risk aversion. Additionally, Pagano and Zechner raise the theory that smaller companies were able to raise cash more easily due to the claim, that loan guarantee programs favored small firms. However, there is no indication that the analyzed sample of small and medium sized enterprises was representative of any specific industry, nor of aid recipients in general \parencite{oced_one_2021}. 
The disproportionate effect on SMEs is related to the overrepresentation of SMEs in industries that are particularly affected by the pandemic shock and the tendency to have smaller cash reserves than larger companies 


Credits from banks, if obtainable, increase the firm's leverage and could make the firm vulnerable to new liquidity shortfalls. And additional leverage only prevents insolvency if there is a prospect that future cash flows will enable a firm to service the additional debt. Regarding solvency, increased leverage means a weaker equity ratio. An early Survey study from September 2020 analyzed the implications of the pandemic crisis on the equity of German companies and reported that for most companies the equity ratio did not change, however a strong sectoral heterogeneity with travel and gastronomy having a reduction in the equity ratio between 1.8 \% and 1.5 \% \parencite{peichl_eigenkapitalentwicklung_2021}. In Spain a survey looked at the indebtedness as well as the cash ratio of enterprises and reported findings that support the heterogeneity of the covid 19 shock across firms and, that the impact was larger for small, young and less productive firms located in urban areas \parencite{fernandez-cerezo_firm-level_2021}. Further support for the heterogeneity of the impact of the COVID-19 crisis on firms’ sales and costs came from Belgium \parencite{dhyne_belgian_2021}. Regarding the indebtedness of firms, \parencite{julin_firm_2021} report for Denmark that credit growth was modest during the pandemic and that firms even find managed to reduce.

The effects of increased debt can be versatile in many aspects and depend on various factors. A simulation on 14 relatively well-covered European countries estimated that an increase in the financial debt of companies has on average a negative impact on the growth of investment after the crisis, indicating negative long-term effects of increased leverage \parencite{demmou_insolvency_2021}.



%----------------------------------------------------------------------------------------
%	SECTION 2
%----------------------------------------------------------------------------------------

\section{Government support effects}

The magnitude of policy responses has already provoked many researchers to look into the effects and effectiveness of various support measures. The difficult data situation has led scientists to explore different routes. 
Early attempts overcame the lack of data by conducting simulation studies. \parencite{ebeke_corporate_2021} estimate that in Europe the share of illiquid firms would have tripled from pre-crisis levels in the absence of policy measures.
A modeling approach by \parencite{puhr_have_2021} indicates that supporting measures in Austria helped to reduce insolvencies by around one-third. 
For European companies, a simulation by \parencite{demmou_liquidity_2021} suggests that the combination of different measures helped to reduce the share of illiquid companies significantly with relief for wage bills being the most effective tool.

A model by \parencite{chang_studying_2022} suggests that deferring taxes is the single best option for income cuts of 25 \%, but a combination of loan and equity based aid is the best option when revenue drops are larger than 50 \%.
The modeling by \parencite{parlapiano_effects_2020} supports the effectiveness of Italian support measures in reducing illiquidity, but also reports that loan based aid increased the indebtedness measured by a debt-to-asset like ratio of 1.2 \%.

With a conceptual approach \parencite{bischof_bedeutung_2021} assessed the regulatory design of grants in Germany and argue for heterogenous effects for different industries based on their cost structure. Their justification is based on transmission factors which are referring to the relationship between the average decrease in revenue and the average decrease in profit of an industry. The factor can be crucial for the effect of aid schemes that are compensating costs proportionately because it has implications on the relative compensation of profits \parencite{bischof_bedeutung_2021}. According to the numbers provided in the paper, Food and beverage service activities are industries with a higher transmission than Travel agency and tour operator activities, as well as creative, arts and entertainment activities. Even lower transmission is reported for the industry of sports activities and amusement and recreation activities.



Other assessments were based on survey data from various countries in many ways. Indications for positive effects aid for micro microenterprises as well as self-employed are reported by \parencite{kochaniak_effectiveness_2023,bertschek_german_2022}.

A firm-level assessment was conducted by \parencite{bellucci_consequences_2022} suggesting that remedy measures in Italy have almost halved the percentage of illiquid SMEs at the end of 2020. 
In Slovakia, government wage subsidies reduced the probability of illiquidity for recipients \parencite{lalinsky_distribution_2021}.
Similar findings were also reported for Euro area firms by \parencite{de_santis_impact_2021} and worldwide by \parencite{igan_shot_2023}.

\parencite{harasztosi_firm-level_2022} find evidence that companies that got support expanded their balance sheet more than unsupported firms. In case of an expansion through debt, part of the change could be explained by the loan-based aid, but not in the case of equity, although the authors report that policy support raised the probability of an increase in the equity base \parencite{harasztosi_firm-level_2022}.

\parencite{stien_covid-19_2022} analyzed the effect of tax deferrals in Norway and reports evidence for a significant decrease in the risk of bankruptcy.
The measures taken by the Belgian authorities to mitigate the impact of the pandemic on companies' earnings have been effective in averting serious solvency issues \parencite{piette_how_2022}.
\parencite{costa_state-aids_2021} used a difference-in-differences method to assess different types of support in Portugal. The reported average treatment effects for debt-based that aid measures suggest a considerable contribution to firms' liquidity.

\parencite{marek_impact_2022} investigated the effects of a grant based support scheme (November-December aid) in Germany based on survey data for around 2,300 firms. Their study suggests that those who were granted aid have a 5 - 6\% lower probability of having a liquidity shortfall within one month. Their coefficients for longer periods are indicating a decrease in the aid effect on liquidity. Although most estimates are statistically insignificant, the results can mean that liquidity injection can only be measured in the very short run and is vanishing over the following month \parencite{marek_impact_2022} also find strong evidence that firms consider bank loans to be a substitute for the provision of transfers.

\parencite{dorr_small_2022} have a critical view and argue that liquidity support measures in combination with the suspensions of the duty to file for insolvency have caused an insolvency gap, especially for smaller firms. Their understanding of an insolvency gap refers to keeping firms alive that were already struggling before the pandemic.

