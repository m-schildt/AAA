% Chapter Template

\chapter{Methods} % Main chapter title

\label{Chapter4} % Change X to a consecutive number; for referencing this chapter elsewhere, use \ref{ChapterX}


%----------------------------------------------------------------------------------------
%	SECTION 1
%----------------------------------------------------------------------------------------

\section{Balance Sheet Ratios}
\label{section:BSratios}

To evaluate the financial position and performance of firms in a comparable way across the data set a selection of balance sheet ratios were chosen. Ratios allow a consistent view on the companies despite their different sizes. Even though balance sheets only offer a reporting date view on the firm's financial information and can't reflect events or extreme situations during a fiscal year, they provide comparable view on companies that is standardized by accounting standards.  The selection of ratios was made to get a picture of the liquidity and solvency the of firms. The ratios are calculated for each beneficiary of government support for each available year between 2018 and 2021. Calculations are shown in table \ref{tab:RatioCalc}.

\subsection{Liquidity Ratios}

Liquidity ratios are chosen to measure a firm's financial position to meet its obligations in the short run. As outlined in chapter \ref{Chapter1}the pandemic shock had a significant effect on companies' liquidity and was a key consideration for the EU to loosen state aid regulation and enable large scale support measures \parencite{eu_com_temporary_2020}. The first and most representative liquidity ratio is the cash ratio, comparing the most liquid asset, cash holdings, to the total assets of a firm. Cash is the starting buffer against running costs in a crisis shock. Although usually the current liabilities are used instead of the total assets, with the available data total assets serve as a more robust denominator that has been utilizes in similar research \parencite{fernandez-cerezo_firm-level_2021, costa_state-aids_2021,igan_shot_2023}. The quick and the current ratio provide a more conservative view on a firms liquidity by including assets that are still considered relatively liquid against the current (short-term) liabilities. However, the key Component is short term debt, is not disclosed consistently in balance sheets due to the fact that accounting standards allow alternative discloser in the balance sheet appendix. For practicability, such a case the calculation had to use the total liabilities, which reduced the informative value in comparisons across firms, but still allows for comparisons on a firm level between different years.

\subsection{Solvency Ratios}

The other factor of interest is the indebtedness of firm in context with the pandemic and the remedy measures. The indebtedness, or also leverage, of a firm has implications that are rather relevant in the long-term, since debt payments are long term obligations that need to be serviced by cash flows. High levels of debt can challenge a company and can reduce profits. 
The debt-to-asset and the equity ratio compare the respective capital to the total assets and are behaving in opposite directions. The debt-to-equity ratio give a magnified picture on the companies leverage compared to the debt-to-asset ratio. For the simplification purposes negative ratios were omitted since result either from errors in the data parsing process or from exceptional cases like loss transfer agreements with parent companies. 

\begin{table}%[]
    \caption{The calculation of Balance Sheet Ratios.}
    \label{tab:RatioCalc}
    \centering
    \def\arraystretch{1.5}
    \begin{tabular}{@{}lll@{}}
    \toprule
    Category                   & Ratio                & Calculation \\ \midrule
    \multirow{3}{*}{Liquidity} & Cash Ratio           & $\frac{Cash}{Total Assets}$ \\ %\cmidrule(l){2-3} 
                                & Quick Ratio          & $\frac{Current Assets-Inventory}{Current Liabilities}$ \\ %\cmidrule(l){2-3} 
                                & Current Ratio        & $\frac{Current Assets}{Current Liabilities}$ \\ \midrule
    \multirow{3}{*}{Liability} & Debt-to-Equity Ratio & $\frac{Debt}{Equity}$ \\ %\cmidrule(l){2-3} 
                                & Equity Ratio         & $\frac{Equity}{Total Assets}$ \\ %\cmidrule(l){2-3} 
                                & Debt-to-Assets Ratio & $\frac{Debt}{Total Assets}$ \\ \bottomrule
    \end{tabular}
\end{table}








%----------------------------------------------------------------------------------------
%	SECTION 2
%----------------------------------------------------------------------------------------

\section{Difference-in-Differences}

With the obtained firm-level data the first analysis tries to (1) measure the causal treatment effect of government support during the COVID-19 pandemic and (2) explore how the effects between aid instruments on the ratios from section \ref{section:BSratios} differ. To estimate the causal effect of aid, the fact the aid measures were granted consecutively over the years 2020-2023 is used for a natural experiment with a standard difference-in-differences method to estimate the average treatment effect on the treated ($ATT$). In this setting the $ATT$ can be described as the average causal effect of an aid instrument on the balance sheet ratio of companies that received support. In mathematical terms can be described as following:

\begin{equation}
    ATT = E[Y_{ratio,1i}| aid=1]-E[Y_{ratio,0i}| aid=1] 
    \label{eqn:ATT}
\end{equation}

The first term describes the expected balance sheet ratio amongst the companies that got aid. The second part describes the unobservable expected ratios of the very same group of companies if they won't have received any aid. In the quasi-experimental setup, the periods 2019 and 2020 will be compared and companies that received support in 2020 serve as treated group, while companies that did not receive support in 2020, but later in 2021 or 2022 serve as the control group. The classification in treated and untreated is based on the cut-off dates of the firms of balance sheets and the date of granting the support. The setting is also performed for the periods 2020 and 2021, where the companies in the control group only received aid in 2022. For the estimation of the difference-in-difference a regression with the following linear model is used:


\begin{equation}
    ratio_{ft} = \beta_{0} + \beta_{1}aid_{f} + \beta_{2}post_{t} + \beta_{3}aid\ast post_{ft} + \varepsilon_{ft} 
    \label{eqn:Diff&Diff}
\end{equation}

Where the dependent variable is the ratio for firm ($f$) in period ($t$). The coefficients on the right side of the equation are the dummy variables aid, post, their interaction term $ aid\ast post$ and the unobserved "error" term $ \varepsilon$. The first independent variable $aid$ equals 0 when firm ($f$) did not receive support until after the experiment period and is in the control group. The variable $aid$ equals 1 when the firm did receive support and is considered treated in experiment period and. The second independent variable $post$ indicates pre- and post-shock periods resembled by 0 and 1. The third and main variable for the difference-in-difference method is the interaction term $ aid\ast post$ which will only be 1 for a treated firm ($f=1$) in the treatment period ($t=1$). The coefficient of the interaction describes the change in the dependent variable due to the treatment as illustrated in a table \ref{tab:RatioCalc}. It can be seen that by inserting the values of aid and post the coefficient $\beta_{3}$ of the interaction of $aid$ and $post$ is the estimated difference-in-difference.

\begin{table}%[]
\caption{Difference-in-difference with regression}
\label{tab:DiDcoefficient}
\centering
\def\arraystretch{1.5}

\begin{tabular}{l|l|l|l|}
    \cline{2-4}
                                            & After($post=1$) & Before ($post=0$)& After - Before \\ \hline
    \multicolumn{1}{|l|}{Treated ($aid=1$)}           &  $\beta_{0}+\beta_{1}+\beta_{2}+\beta_{3}$     &  $\beta_{0}+\beta_{1}$      &   $\beta_{2}+\beta_{3}$             \\ \hline
    \multicolumn{1}{|l|}{Control ($aid=0$)}           &  $\beta_{0}+\beta_{2}$     &  $\beta_{0}$      &   $\beta_{2}$             \\ \hline
    \multicolumn{1}{|l|}{Treated - Control} &  $\beta_{1}+\beta_{3}$     &  $\beta_{1}$      &   $\beta_{3}$             \\ \hline
    \end{tabular}

\end{table}

The central assumption for the difference-in-difference methodology is that the control and the treatment group are comparable and would behave parallel in over the observed periods, if there wouldn't be a treatment. The proposed setup is based on the assumption that companies who received aid at any point during the COVID-19 pandemic were sufficiently affected from the shock to be eligible for support and subsequently confirm the parallel trends assumption.

The methodology will be used to measure the effects of grants and loans separately for all ratios in 2020 as well as 2021. Due to the outliers in the ratios the upper and lower fifth percentile is omitted. 







%----------------------------------------------------------------------------------------
%	SECTION 3
%----------------------------------------------------------------------------------------

\section{Causal Curve}

Next, the effect of support during the COVID-19 pandemic will be further analyzed by considering the amount of support as a continuous variable, instead of a binary variable like in the difference-in-differences approach. Estimating the effect of a continuous treatment can be evaluated through generalized propensity scores and described by a dose-response function showing the causal effect at each dose of treatment. 

The groundwork was done by \parencite{hirano_propensity_2004}, by exploring the propensity scores methods for continuous treatments. 

Generally, propensity scores refer to the probability of a unit being exposure to a treatment (binary or continuous) given pre-exposure covariates \parencite{rosenbaum_central_1983}. 

The initial approach by Hirano and Imbens estimates the generalized propensity score (GPS) as the probability density function of the treatment conditioned on pre-exposure covariates, before combining the GPS with the treatment in a regression model to estimate a response to a treatment dose \parencite{hirano_propensity_2004}.

Compared to a traditional regression method, the GPS approach allows adjustment for possible confounders and for removal of bias in the estimation of the treatment effect, which can be crucial in observational studies \parencite{wu_matching_2021,moodie_estimation_2012}. 

Efforts by \parencite{moodie_estimation_2012} and \parencite{galagate_causal_2016} expanded on the concept of GPS' from Hirano and Imbens and proposed alternative estimators for a dose-response functions.



This thesis uses the concept of transforming continuous treatments into probability density functions in order to approximate a dose-response for the government support. The estimation for both the generalized propensity scores and the dose-response function follows an implementation from \parencite{kobrosly_causal-curve_2020}. 
The approach consists of three steps. First, the probability of the treatment T, referring to the amount of aid, is modelled by conditioning it on a set of covariates X using a normal generalized linear model (GLM):


\begin{equation}
    T_{i}|X_{I}\sim \mathcal{N} (X^{T}_{i}\beta,\sigma^{2})
    \label{eqn:densit}
\end{equation}

The probability density function of the fitted values and the actual treatment is then estimated to get a GPS for each firm. To prevent possible confounding the variable shown in \ref{tab:Covariates} were used as covariates. 
In the second step the outcome Y, representing the observed change in liquidity for each firm, is fitted on the observed treatment T and the GPS from the first step using a linear generalized additive model. 

\begin{table}%[]
    \caption{Covariates for GPS and dose-response function}
    \label{tab:Covariates}
    \centering
    \def\arraystretch{1.2}
    \begin{tabular}{ll}
        \hline
        Covariates                 & Explanation                                                                                                                                     \\ \hline
        aid\_loan\_2020            & \begin{tabular}[c]{@{}l@{}}Values of loans that the company \\ received in 2020\end{tabular}                                                    \\
        aid\_loan\_2021            & \begin{tabular}[c]{@{}l@{}}Values of loans that the company \\ received in 2021\end{tabular}                                                    \\
        aid\_loan\_2020            & \begin{tabular}[c]{@{}l@{}}Values of grants that the company \\ received in 2020\end{tabular}                                                   \\
        assets\_2020               & \begin{tabular}[c]{@{}l@{}}Total assets of the company in 2020 \\ as a proxy for company size\end{tabular}                                      \\
        debt\_to\_asset ratio 2020 & \begin{tabular}[c]{@{}l@{}}Ratio to reflect the indebtedness of \\ a company prior the aid as granted\end{tabular}                              \\
        days\_grant\_2021          & \begin{tabular}[c]{@{}l@{}}Variable that indicates the days between \\ the granting and the cutoff date of the \\ financial report\end{tabular} \\ \hline
        \end{tabular}
        \end{table}

    In the last step the model is used to predict the treatment response in liquidity at a multiple levels to of the aid payments to visualize the causal effect as a causal dose response curve (CDRC).

    The needed assumptions for the methods are the government support for one company will not affect the outcome for another company (Stable Unit Treatment Value Assumption). The only scenario where this could be violated is, when two affiliated companies in the data set both received support but pool the money in one company. Second, the positivity assumption can also be confirmed since every company has some possibility of receiving support and the support is a positive. The third assumption of unconfoundedness is most critical for observational studies. It has to be assumed that any relationship between potential outcomes in liquidity and the received support can be fully explained by the included covariates. By including the in table \ref{tab:Covariates} presented covariates the issue of potential unobserved confounders was attempted to reduces best possible with the available data. How, it is likely that there are additional unmeasured confounders that that weren’t observed introduced bias to the results. 
        