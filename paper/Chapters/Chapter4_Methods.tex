% Chapter Template

\chapter{Methods} % Main chapter title

\label{Chapter4} % Change X to a consecutive number; for referencing this chapter elsewhere, use \ref{ChapterX}

%----------------------------------------------------------------------------------------
%	SECTION 1
%----------------------------------------------------------------------------------------

\section{KPI}

Research unambiguous concluded that COVID-19 crisis negatively influenced the economy in countries around the world. Many businesses were severely affected by drops in demand and lockdowns by the authorities. The pandemic shock leads to negative cash flows for many firms (Fernández-Cerezo et al. 2021). Depending on the affectedness of the business and the cash holding, liquidity shortfalls are inescapable. 
Without continuation of their business and positive cash flows, firm’s equity and the liquidity (cash and bank) positions will inevitably deteriorate. At some point, firms are in need of Liquidity injection, either through additional equity or via debt. However, debt, if obtainable, increases the firms leverage and could make the firm vulnerable to new liquidity shortfalls. And, additional leverage only prevents from insolvency if there is a prospect that future cash flows will enable a firm to service the additional debt.
The effect of the COVID-19 outbreak is widely described as an economic shock,

Pagano and Zechner empirically analyzed the effects of covid 19 on companies’ financial performance in the EU (2022). Their finding shows significant differences in the effects between large firms and small and medium sized enterprises. 
By comparing the year 2019 and 2020 the authors found that smaller companies tend to increase their ratio of total debt to total assets (debt-ratio) whereas, large companies also increase their leverage, but significantly less.
Regarding liquidity, small and medium sized enterprises increased their cash to total assets ratio more than large companies. Small companies did so even more than medium sized ones. However, the authors could only speculate over the reason behind of this observation. Plausible reasons were precautionary cash hording and greater risk aversion. Additionally, the authors raise the theory that smaller companies were able to raise cash more easily due to the claim, that loan guarantee programs favored small firms. However, the analyzed sample of small and medium sized enterprises was not representative of any specific industry, nor of aid recipients. 
A study by Peichl et al. analyses the implications of the pandemic crisis on the equity of Germany companies (2021). They found in their early survey from September that for most companies the equity ratio did not change, however a strong sectoral heterogeneity with travel and gastronomy having a reduction in the equity ratio between 1.8 % and 1.5 %.
(Tielens et al. 2021) conducted a significant short-run impact on firms’ liquidity buffers in Belgium by the covid 19 Pandemic. (Narrow liquidity ratio) And heterogeneous impact of the COVID-19 crisis on the cash position of Belgian firms in comparison to a business-as-usual counterfactual.
In Spain a survey looked at amongst other indicators looked at indebtedness and cash ratio of enterprises. Findings support the heterogeneity of the covid 19 shock across firms and that the impact was larger for small, young and less productive firms located in urban areas. (Fernández-Cerezo et al. 2021).
Heterogeneous impact of the COVID-19 crisis on firms’ sales and costs, see Dhyne and Duprez (2021)




%----------------------------------------------------------------------------------------
%	SECTION 2
%----------------------------------------------------------------------------------------

\section{Diff and Diff}

Policy intervention to "prevent” the effects and save businesses for a fast economic recovery.
First assessments were modeling approaches.
Already lots of early assessments of state aid, also at a firm level.
For getting a better understanding on the effect of aid schemes in Germany a paper analyses the effect of a company’s cost structure on the effectives of aid measures (Bischof, Karlsson, Rostam-Aschar, Simon, 2021). 
This paper assumes that companies within the same sector have a similar cost structure. 
Since aid in Germany is based on the cost structure of companies, the authors conclude that based on the generalized approach of aid schemes, the effectiveness of aid is varying between business sectors.


%----------------------------------------------------------------------------------------
%	SECTION 3
%----------------------------------------------------------------------------------------

\section{Causal Curve}

Policy intervention to "prevent” the effects and save businesses for a fast economic recovery.
First assessments were modeling approaches.
Already lots of early assessments of state aid, also at a firm level.
For getting a better understanding on the effect of aid schemes in Germany a paper analyses the effect of a company’s cost structure on the effectives of aid measures (Bischof, Karlsson, Rostam-Aschar, Simon, 2021). 
This paper assumes that companies within the same sector have a similar cost structure. 
Since aid in Germany is based on the cost structure of companies, the authors conclude that based on the generalized approach of aid schemes, the effectiveness of aid is varying between business sectors.