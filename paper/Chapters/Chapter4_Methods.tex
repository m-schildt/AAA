% Chapter Template

\chapter{Methods} % Main chapter title

\label{Chapter4} % Change X to a consecutive number; for referencing this chapter elsewhere, use \ref{ChapterX}

%----------------------------------------------------------------------------------------
%	SECTION 1
%----------------------------------------------------------------------------------------

\section{Balance Sheet Ratios}

To evaluate the financial position and performance of firms in a comparable way across the data set a selection of balance sheet ratios were chosen. Ratios allow a consistent view on the companies despite their different sizes. Even though balance sheets only offer a reporting date view on the firm’s financial information and can’t reflect events or extreme situations during a fiscal year, they provide comparable view on companies that is standardized by accounting standards.  The selection of ratios was made to get a picture of the liquidity and solvency the of firms. The ratios are calculated for each beneficiary of government support for each available year between 2018 and 2022. Calculations are shown in table \ref{tab:RatioCalc}.

\subsection{Liquidity Ratios}

Liquidity ratios are chosen to measure a firm’s financial position to meet its obligations in the short run. As outlined in chapter \ref{Chapter1}the pandemic shock had a significant effect on companies’ liquidity and was a key consideration for the EU to loosen state aid regulation and enable large scale support measures \parencite{eu_com_temporary_2020}. The first and most representative liquidity ratio is the cash ratio, comparing the most liquid asset, cash holdings, to the total assets of a firm. Cash is the starting buffer against running costs in a crisis shock. Although usually the current liabilities are used instead of the total assets, with the available data total assets serve as a more robust denominator that has been utilizes in similar research \parencite{fernandez-cerezo_firm-level_2021, costa_state-aids_2021,igan_shot_2023}. The quick and the current ratio provide a more conservative view on a firms liquidity by including assets that are still considered relatively liquid against the current (short-term) liabilities. However, the key Component is short term debt, is not disclosed consistently in balance sheets due to the fact that accounting standards allow alternative discloser in the balance sheet appendix. For practicability, such a case the calculation had to use the total liabilities, which reduced the informative value in comparisons across firms, but still allows for comparisons on a firm level between different years.

\subsection{Solvency Ratios}

The other factor of interest is the indebtedness of firm in context with the pandemic and the remedy measures. The indebtedness, or also leverage, of a firm has implications that are rather relevant in the long-term, since debt payments are long term obligations that need to be serviced by cash flows. High levels of debt can challenge a company and can reduce profits. 
The debt-to-asset and the equity ratio compare the respective capital to the total assets and are behaving in opposite directions. The debt-to-equity ratio give a magnified picture on the companies leverage compared to the debt-to-asset ratio. For the simplification purposes negative ratios were omitted since result either from errors in the data parsing process or from exceptional cases like loss transfer agreements with parent companies. 

\begin{table}[]
    \caption{The calculation of Balance Sheet Ratios.}
    \label{tab:RatioCalc}
    \centering
    \def\arraystretch{1.5}
    \begin{tabular}{@{}lll@{}}
    \toprule
    Category                   & Ratio                & Calculation \\ \midrule
    \multirow{3}{*}{Liquidity} & Cash Ratio           & $\frac{Cash}{Total Assets}$ \\ %\cmidrule(l){2-3} 
                                & Quick Ratio          & $\frac{Current Assets-Inventory}{Current Liabilities}$ \\ %\cmidrule(l){2-3} 
                                & Current Ratio        & $\frac{Current Assets}{Current Liabilities}$ \\ \midrule
    \multirow{3}{*}{Liability} & Debt-to-Equity Ratio & $\frac{Debt}{Equity}$ \\ %\cmidrule(l){2-3} 
                                & Equity Ratio         & $\frac{Equity}{Total Assets}$ \\ %\cmidrule(l){2-3} 
                                & Debt-to-Assets Ratio & $\frac{Debt}{Total Assets}$ \\ \bottomrule
    \end{tabular}
    \end{table}

%----------------------------------------------------------------------------------------
%	SECTION 2
%----------------------------------------------------------------------------------------

\section{Diff and Diff}

Policy intervention to "prevent” the effects and save businesses for a fast economic recovery.
First assessments were modeling approaches.
Already lots of early assessments of state aid, also at a firm level.
For getting a better understanding on the effect of aid schemes in Germany a paper analyses the effect of a company’s cost structure on the effectives of aid measures (Bischof, Karlsson, Rostam-Aschar, Simon, 2021). 
This paper assumes that companies within the same sector have a similar cost structure. 
Since aid in Germany is based on the cost structure of companies, the authors conclude that based on the generalized approach of aid schemes, the effectiveness of aid is varying between business sectors.


%----------------------------------------------------------------------------------------
%	SECTION 3
%----------------------------------------------------------------------------------------

\section{Causal Curve}

Policy intervention to "prevent” the effects and save businesses for a fast economic recovery.
First assessments were modeling approaches.
Already lots of early assessments of state aid, also at a firm level.
For getting a better understanding on the effect of aid schemes in Germany a paper analyses the effect of a company’s cost structure on the effectives of aid measures (Bischof, Karlsson, Rostam-Aschar, Simon, 2021). 
This paper assumes that companies within the same sector have a similar cost structure. 
Since aid in Germany is based on the cost structure of companies, the authors conclude that based on the generalized approach of aid schemes, the effectiveness of aid is varying between business sectors.