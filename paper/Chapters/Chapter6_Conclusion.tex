% Chapter Template

\chapter{Conclusion} % Main chapter title

\label{Chapter6} % Change X to a consecutive number; for referencing this chapter elsewhere, use \ref{ChapterX}

%----------------------------------------------------------------------------------------
%	SECTION 1
%----------------------------------------------------------------------------------------

\section{Policy Implications}

To mitigate the effects of the pandemic shock the German government responded with unprecedented fiscal efforts to provide liquidity to the economy and prevent business from failing. Policy responses of this magnitude are complex in many ways. The regulative aspect regarding European state aid regulations has been covered shortly in the introduction.

Finding the right balance between the effectiveness and efficiency of the policy response is a key challenge. On the one hand the support needs to be appropriate to provide sufficient relief since an economic collapse is not a considerable alternative. On the other hand, support needs also be targeted, helping with liquidity where needed and not creating excessive benefits. 
Under the pressure to keep the economy creating targeted aid measures that adequate aid for everyone is hardly obtainable.

For example, the right level of assistance is very crucial for highly vulnerable companies that were already struggling before the pandemic shock. Legitimate concerns about the side effects of generous support to this group of companies are present. The zombification, which refers to artificially keeping these companies alive and delaying inevitable default, as well as the distortion of competition are part of the concerns \parencite{dorr_small_2022}. But also, considerations of fairness and the interpretation to whom the fairness should be interpreted. Policy makers were faced with difficult tradeoffs, especially given the possible chaining on insolvencies, which could have unmanageable consequences and possibly counteract the actual goals of the aid and hamper the effort that was already made.

A method that was utilized through grants schemes were provisional permits with expedited payouts for quick relief, but subject to repayment. Thus, allowing for a final determination of the granted amount at a later date. 

In regard to the assessment of support measures no judgment on observed excessive liquidity is possible, since funds could still be due for repayment.




%----------------------------------------------------------------------------------------
%	SECTION 2
%----------------------------------------------------------------------------------------

\section{Conclusion}

With the DiD regression effects from aid on the liquidity and solvency of companies were successfully measured. The results suggest that in 2020 loan-based measures supported companies by improving their liquidity by 5.3 \%, while grant-based aid was insufficient as a liquidity injection since beneficiaries had no improvement in liquidity and relatively higher debt. However, in 2021 grants programs became effective in providing liquidity and preserving the equity of firms. In 2021 the liquidity increase of grant beneficiaries was 7.7 \%, twice as high as the effect of grants in 2021. 

Moreover, the insolvencies of aid beneficiaries were analyzed. Results suggest that they were already significantly weaker before the pandemic and that aid measures were less effective in supporting them compared to the overall population of beneficiaries. However, a sector view reveals that the industries that are most represented amongst beneficiaries gastronomy and accommodation show insolvency rates well below the average, suggesting that aid measures were successful in reducing the chance of insolvency in their presumed main target sectors.

Finally, the thesis provided a deeper understanding of the aid measures by exploiting the data granularity with a combination of generalized propensity scores and a generalized linear model (GLM). The visualizations were showing the relationship between changes in beneficiaries' liquidity as well as solvency at different aid levels. Overall, the visualizations are in line with the results from the DiD and don't show any concerning abnormalities. In greater detail, the visualizations reveal some heterogeneity amongst beneficiaries from different industries, which connects with the observed heterogeneity in insolvencies of different industries.

With the results obtained, limitations must also be recognized. In addition to limitations arising from the data sources and the inherent risk of mismatches during that matching process, the causal modeling for the DiD and the GPS involved strong assumptions such as the assumption of parallel trends and unconfoundedness that, if violated, can heavily bias the estimates.
 
In consideration of the overall results of the thesis and already existing contributions from other researchers the government support had a considerable role in carrying the economy through the pandemic. However, the future will require more in-depth assessments that fully reflect the repayments of excessive liquidity support and have a longer-term view on beneficiaries' insolvencies.
