% Chapter Template

\chapter{Conclusion} % Main chapter title

\label{Chapter6} % Change X to a consecutive number; for referencing this chapter elsewhere, use \ref{ChapterX}

%----------------------------------------------------------------------------------------
%	SECTION 1
%----------------------------------------------------------------------------------------

\section{Policy Implications}

To mitigate the effects of the pandemic shock the German government responded with unprecedented fiscal efforts to provide liquidity to the economy and prevent business from failing. Policy responses in this magnitude are complex in many ways. The regulative aspect regarding European state aid regulations has been covered shortly in the introduction.

Finding the right balance between effectiveness and efficiency of the policy response a key challenge. On the one hand the support needs to be appropriate to provide sufficient relief since an economic collapse is not a considerable alternative. On the other hand, support needs also be target, helping with liquidity were needed and not created excessive benefits. 
Under the pressure to keep the economy creating targeted aid measures that adequate aid for everyone is hardly obtainable.

For example, the right level of assistance is very crucial for highly vulnerable companies that were already struggling before the pandemic shock. Legitimate concerns about side effects of generous support to the group up companies are present. The zombification, which refers to artificially keeping these companies alive and delaying inevitable default, as well as the distortion of competition are part of the concerns \parencite{dorr_small_2022}. But also, considerations of fairness and the interpretation to whom the fairness should be interpreted. Policy makers were faced with difficult tradeoffs, especially in view of a possible chaining on insolvencies, which could have unmanageable consequences and possibly counteract the actual goals of the aid and hamper the effort that were already made.

A method that was utilized through grants schemes were provisional permits with expedited payouts for quick relief, but subject to repayment. Thus, allowing for a final determination of the granted amount at later date. 

In regard to the assessment of support measures no judgement on observed excessive liquidity is possible, since funds could still be due for repayment.




%----------------------------------------------------------------------------------------
%	SECTION 2
%----------------------------------------------------------------------------------------

\section{Conclusion}

