% Chapter Template

\chapter{Results} % Main chapter title

\label{Chapter5} % Change X to a consecutive number; for referencing this chapter elsewhere, use \ref{ChapterX}

%----------------------------------------------------------------------------------------
%	SECTION 1
%----------------------------------------------------------------------------------------

\section{Data insights on beneficiaries}
\label{BSratio}

\subsection{Liquidity}

In Figure \ref{fig:Ratios} the calculated ratios for all companies of the dataset are shown with boxplots for the years 2018-202. With the median, shown as the middle line in the box, the first and the third quartile, boxplots allow a simple comparison between the years that is not sensitive for outliers. The three observed liquidity ratios show an increase in liquidity in 2020 and 2021 compared to the pre-pandemic years, indicating that companies were holding relatively more cash at the year-end since the pandemic. 
Overall, the observations are in line with a study conducted by the German Federal Bank that also reported an increase in the average cash ratio for the whole population of German companies in 2020 as well as in 2021 \parencite{deutsche_bundesbank_jahresabschlussstatistik_2022}. For example, for SME corporations the study reported a change in the cash ratio from 0.104 (2019) to 0.110 (2020). For the current ratio and quick ratio, the same trend was reported. Further support for an increase in the quick ratio comes from another study by \parencite{bley_mittelstand_2022}. Although the exact ratios are varying between studies, there is strong support for the general trend of increasing liquidity in 2020 and 2021.
What also can be seen in Figure \ref{fig:Ratios} is that the whiskers of the boxplots are longer in the pandemic years, showing the heterogeneity of liquidity levels for firms. Considering that the ratios only reflect the status of the cutoff date, this can also be an indicator that the liquidity levels of firms are more volatile than before the pandemic which could be caused by forecasting difficulties in the liquidity planning during the pandemic. 


\subsection{Solvency}

Solvency ratios are showing a less clear trend after the COVID-19 pandemic. Although minimal, the opposite trends in the equity ratio and debt-to-asset ratio are as expected. The only visible change happened in 2020, while in 2021 the ratios are very similar to 2018 and 2019. The change in the debt-to-asset ratio is amplified in the debt-to-equity ratio, as expected. Survey Data from the KFW found an Equity Ratio of 0.318 in 2019, a decrease to 0.301 in 2020, and a recovery to 0.314 in 2021 \parencite{kfw_kfw-mittelstandspanel_2022}. For very small companies with less than 10 employees, the drop in 2020 was stronger, and the recovery in 2021 was above pre-pandemic levels. On the other hand, the survey reported
that larger companies did not have a recovery after the first crisis year and decreased their Equity Ratio in 2021 on average further. 
This could indicate that the recovery of the indebtedness of beneficiaries in 2021 might have been driven by smaller companies. Similar observations were reported by the German Federal Bank were the debt-to-asset ratio for SME corporations decreased in 2020 \parencite{deutsche_bundesbank_jahresabschlussstatistik_2022}.



\begin{figure}
\centering
\makebox[\textwidth][c]{\includegraphics[width=1.3\columnwidth]{Figures/chart_ratios}}%

\decoRule
\caption[Indicators of liquidity and solvency 2019-2021]{Boxplots with balance sheet ratios from the obtained dataset. Extreme outliers above the maximum values are not shown.}
\label{fig:Ratios}
\end{figure}






\subsection{Insolvencies of beneficiaries}


In total 953 insolvent firms were identified representing a share of 0.92 \% of all beneficiaries of pandemic aid from the transparency database, as shown in Table \ref{tab:InsBySize}. A further breakdown reveals that the share of SMEs is significantly lower than their larger counterparts. Since the transparency database only provides very few beneficiaries with aid payments below 100.000 EUR, smaller companies with aid below the threshold are not well represented. Therefore, the numbers only give an indication.

\begin{table}
\caption{Share of insolvent aid  beneficiaries by size}
\label{tab:InsBySize}
\centering

\def\arraystretch{1.2}
\centering
\begin{tabular}{lrrl}
\toprule
           size &  aid beneficiaries &  insolvent & share \\
\midrule
           SMEs &              78077 &        526 & 0.67\% \\
Large companies &              25259 &        427 & 1.69\% \\
          Total &             103336 &        953 & 0.92\% \\
\bottomrule
\end{tabular}
}

\end{table}

In the next step the industries of insolvent beneficiaries were analyzed. The results for the most represented industries are presented in Table \ref{tab:InsByIndustry_short}. A more comprehensive table with more industries is in Appendix \ref{AppendixB}. The findings show that food and beverage service activities are the industry with the most beneficiaries, but only have a share of 0.58 \% of insolvencies that well below the average. 

Similarly, the accommodation sector is highly represented. With only 24 insolvencies out of 9.885 recipients, it has the lowest share amongst the most represented industries. On the higher end are specialized construction activities with a share of 1.66 \%. The closely related construction building sector has an even higher share with 2.97 \%, but is less represented and therefore only shown in the appendix. Partially, even higher shares are present in the data, but only in underrepresented industries. 

\begin{table}
    \caption{Share of insolvent aid  beneficiaries by industry}
    \label{tab:InsByIndustry_short}
    \centering
    \def\arraystretch{1}
    \centering
    \begin{tabular}{lrrl}
\toprule
                                industry &  beneficiaries &  insolvent & share \\
\midrule
    Food and beverage service activities &          15173 &         88 & 0.58\% \\
Retail trade, except of motor vehicles a &           8810 &         84 & 0.95\% \\
     Specialised construction activities &           4103 &         68 & 1.66\% \\
Wholesale trade, except of motor vehicle &           5396 &         56 & 1.04\% \\
Manufacture of fabricated metal products &           3105 &         40 & 1.29\% \\
Office administrative, office support an &           3040 &         28 & 0.92\% \\
Sports activities and amusement and recr &           5232 &         27 & 0.52\% \\
                           Accommodation &           9885 &         24 & 0.24\% \\
Wholesale and retail trade and repair of &           3525 &         17 & 0.48\% \\
\bottomrule
\end{tabular}
}

    \small  Notes: The table shows industries with more than 3000 beneficiaries and is sorted by insolvencies. The industry names are truncated after 40 characters.

\end{table}




\subsection{Ratios of insolvencies}

Next, the ratios of insolvent beneficiaries are analyzed and compared to the other beneficiaries. For the comparison, the cash ratio and the debt-to-asset ratio are chosen to analyze the liquidity and solvency. Figure \ref{fig:RatiosInsolvency} shows the already for section \ref{BSratio} used boxplot, but grouped into a solvent (light blue) and insolvent by the spring of 2023 (darker blue). For the liquidity ratio, a clear discrepancy can be observed throughout all periods. The group of companies that later got insolvent has already before 2020 significantly less liquidity than the other group. However, in 2020 and 2021 their cash ratio is also increasing, like the solvent group.
The debt-to-asset ratio comparison also shows a clear gap between both groups. The companies which are later becoming insolvent have higher leverage before and after the start of the pandemic. 

\begin{figure}
    \centering
    \makebox[\textwidth][c]{\includegraphics[width=1.2\columnwidth]{Figures/chart_ratios_insolvence}}%
    
    \decoRule
    \caption[Liquidity and solvency of insolvent beneficiaries]{Boxplot with balance sheet ratios from the obtained dataset.}
    \label{fig:RatiosInsolvency}
\end{figure}





%----------------------------------------------------------------------------------------
%	SECTION 2
%----------------------------------------------------------------------------------------

\section{The effect of government support}



\subsection{The average effect on firm liquidity}



The interaction terms from the difference-in-differences regressions are shown in Table \ref{tab:DiDresults}. 
The coefficients for the aid and post coefficients are reported in Appendix \ref{AppendixA}. In the first-row cash ratio coefficients are reported for grants and loans. In the 2020 column the effect for grants is the only cash ratio coefficient that is not statistically significant. 

In 2021 the coefficient for grants on the cash ratio was estimated to be 7.68 \% indicating a strong causal effect. The average treatment effect for firms that got aid through loans is positive in 2020 and 2021, but less strong than grants. Also, the effect in 2021 is less strong compared to 2021.

The quick ratio and the current ratio only have statistical significance for loans. At a significance level of 1 \%, only the quick ratio and the current ratio coefficient for 2020 are significant. The treatment effects with 10.85 \% and 12.71 \% are even stronger than for the cash ratio. The effects for the current ratio are even larger, reflecting the proportionality between the ratios. Overall, the results show strong effects of loans on the observed liquidity ratios. For grants an even stronger effect was observed in 2021, but only for the cash ratio, not for the more conservative ratios. However, this could be caused by the alternative calculation of the quick and current ratio that was used for part of the companies. 

In summary both aid measures have a significant effect on the cash ratio indicating a liquidity boost. The measured increase in liquidity through loans indicates that they were not used for refinancing existing debt, but served as liquidity injection as intended by the policy makers.

\subsection{The average effect on solvency of firms}

In the row for the debt-to-equity ratio strongly positive coefficients were observed for loans in 2020 and 2021. Since the debt-to-equity ratio reflects a firm's leverage it is plausible that loans have a causal effect on debt. Regarding the equity ratio in the following row, a negative effect of loans was estimated for 2020 and 2021. For grants, which aren't affecting a firm's debt, no statistically significant effect was observable and thus indicating that grants didn't support the equity of firms. Further support for the increase in leverage through loans comes from the debt-to-asset ratio coefficients. The effects for loans are strong in 2020 and 2021 with 7.15 \% and 5.26 \%. 
For grants, the causal effect is 4,12 \% in 2020 and reversing in 2021 to - 3.02 \%. The positive effect in 2020 is in line with the missing effect on the equity ratio. The negative coefficient in 2021 on the other hand, indicates that grants helped firms to reduce leverage by slightly.

Consequently, the increase in leverage from loans is measurable as expected. However, the estimates does not support the expectation that grants protect the equity of firms. The positive effect of grants on the debt-to-asset ratio and the missing effect on the cash ratio in 2020 suggest that the grants were not sufficient to prevent liquidity shortfalls and therefore companies had to rely on debt from other sources for additional liquidity. The reversal of the effect on the debt-to-asset ratio and the effect on the cash ratio in 2021 suggest that grants only became effective in providing sufficient liquidity and reducing leverage in 2021. However, this could be related to the fact that only relatively few grants were provided in 2020.

\begin{table}
    \caption{Government aid impact on ratios}
    \label{tab:DiDresults}
    \centering
    \def\arraystretch{1.2}
    \centering
    \begin{tabular}{llrr}
\toprule
                     & \textbf{year} &                2020 &                2021 \\
{} & \textbf{ratio} &                     &                     \\
\midrule
\textbf{cash ratio} & \textbf{grant} &    -0.006   (0.521) &    0.093*** (0.000) \\
                     & \textbf{loan} &    0.065*** (0.000) &   0.0388*** (0.000) \\
\textbf{quick ratio} & \textbf{grant} &   -0.0871   (0.457) &    0.0357   (0.853) \\
                     & \textbf{loan} &   0.0905**  (0.029) &      0.2039 (0.094) \\
\textbf{current ratio} & \textbf{grant} &   -0.1636   (0.204) &   -0.0279   (0.895) \\
                     & \textbf{loan} &   0.1518*** (0.001) &      0.2657 (0.072) \\
\textbf{debt to equity ratio} & \textbf{grant} &   0.5661**  (0.041) &   -0.3862   (0.126) \\
                     & \textbf{loan} &   1.0924*** (0.000) &      0.7023 (0.070) \\
\textbf{equity ratio} & \textbf{grant} &   -0.0095   (0.497) &    0.0136   (0.405) \\
                     & \textbf{loan} &  -0.0575*** (0.000) &  -0.0453*** (0.002) \\
\textbf{debt to assest ratio} & \textbf{grant} &   0.0366**  (0.024) &     -0.0319 (0.083) \\
                     & \textbf{loan} &   0.0764*** (0.000) &   0.0558*** (0.002) \\
\bottomrule
\end{tabular}
}

    \small  Notes: Standard errors in parentheses, *** p<0.01, ** p<0.05, * p<0.1

\end{table}




\subsection{Aid effect on companies that later become insolvent}

The group of future insolvent companies was used for a another difference in differences experiment to estimate the treatment effect of aid on the subgroup. Results are shown in Table \ref{tab:DiDresultsInsolvent}. Two coefficients with statistical significance were identified, both in 2020 and for loans. First, the treatment effect of aid was 0.0427 on the cash ratio. The effect is weaker than the 0.527 that was estimated on the complete data set. This can be understood that these companies absorbed the liquidity from loans quicker due to a greater need for liquidity. The other slightly significant coefficient indicates a positive effect on the debt-to-asset ratio of 0.0882 which is also higher than the overall effect observed from all companies.  This shows that despite the loan based aid measures the later insolvent companies had a greater chance of indebtedness, than other companies.
The higher debt-to-asset-ratio could also be an indication that these companies used debt from other sources. In any case, the, additional indebtedness increased the risk of insolvency.



        

\begin{table}
    \caption{Government aid impact on ratios of insolvent beneficiaries}
    \label{tab:DiDresultsInsolvent}
    \centering
    \def\arraystretch{1.2}
    \centering
    \begin{tabular}{llrr}
\toprule
                     & \textbf{year} &                                    2020 &                                    2021 \\
{} & \textbf{instrument} &                                         &                                         \\
\midrule
\multirow{2}{*}{\textbf{cash ratio}} & \textbf{grant} &   -0.038\space\space\space\space(0.605) &   0.1469\space\space\space\space(0.337) \\
                     & \textbf{loan} &             0.0427**\space\space(0.046) &  -0.1302\space\space\space\space(0.394) \\
\cline{1-4}
\multirow{2}{*}{\textbf{quick ratio}} & \textbf{grant} &   0.2732\space\space\space\space(0.645) &   0.4531\space\space\space\space(0.598) \\
                     & \textbf{loan} &   0.1349\space\space\space\space(0.493) &   0.2365\space\space\space\space(0.728) \\
\cline{1-4}
\multirow{2}{*}{\textbf{current ratio}} & \textbf{grant} &   0.2117\space\space\space\space(0.759) &   0.3419\space\space\space\space(0.666) \\
                     & \textbf{loan} &   0.0444\space\space\space\space(0.848) &  -0.0784\space\space\space\space(0.930) \\
\cline{1-4}
\multirow{2}{*}{\textbf{debt to equity ratio}} & \textbf{grant} &  -3.9105\space\space\space\space(0.867) &   1.0318\space\space\space\space(0.588) \\
                     & \textbf{loan} &   2.0475\space\space\space\space(0.262) &   6.8704\space\space\space\space(0.663) \\
\cline{1-4}
\multirow{2}{*}{\textbf{equity ratio}} & \textbf{grant} &   0.0889\space\space\space\space(0.588) &  -0.1444\space\space\space\space(0.405) \\
                     & \textbf{loan} &  -0.0338\space\space\space\space(0.381) &  -0.2779\space\space\space\space(0.467) \\
\cline{1-4}
\multirow{2}{*}{\textbf{debt to assest ratio}} & \textbf{grant} &   0.0797\space\space\space\space(0.650) &  -0.0763\space\space\space\space(0.728) \\
                     & \textbf{loan} &        0.0892*\space\space\space(0.080) &   0.1806\space\space\space\space(0.580) \\
\bottomrule
\end{tabular}
}

    \small  Notes: Standard errors in parentheses, *** p<0.01, ** p<0.05, * p<0.1

\end{table}
    



%----------------------------------------------------------------------------------------
%	SECTION 3
%----------------------------------------------------------------------------------------

\section{Causal Curve}


\subsection{The relationship between liquidity and aid}

The first implementation of the GPS methodology is used to compare the effects of grants in 2021 and loans in 2020 on firm liquidity in both relative (left) and absolute (right) terms. The respective visualizations are shown in Figure \ref{fig:Curve1}. The choice was based on the findings from the previous section and the data richness of these groups. 

The relative illustrations show the modeled relation between the observed change in the cash ratio in dependence on the relative amount of aid in regard to beneficiaries' total assets. While the relative illustrations show the observed change in cash holdings and the aid amount in absolute terms. All illustrations use 95 \% confidence bands. 

Moreover, axis ranges are cut to allow for comparison within the shared ranges in each figure. The full axis ranges significantly different between groups, depending underlying on in the data.

What can be seen is that all curves show a clear upward trend till at least the middle of the observed ranges. Outstanding is the flattening curve in the bottom left in the second half of the range, while the others continue the upward trend. However, the findings could be related to the loan program “KfW-Sonderprogramm” which was bound to pre-pandemic revenue numbers in combination with fixed caps for the loan volumes. A diminishing effect could therefore appear for larger companies with loans that are capped by fixed thresholds. Since revenue can be considered as an alternative proxy for firm size, the diminishing effect could appear in the modeled relative relationship, but not in the absolute perspective.

With respect to the grants in 2021 parallel trends of the absolute and the relative views indicate a positive relationship between the level of aid and the liquidity increase which would be in line with the average causal effect of 7.7 \% measured by the DiD approach. 

The drop in the curve of the relative view on loans in 2020 could be represented by the smaller average effect liquidity (5.3 \%) from the DiD approach Given the strong award trend in the absolute view, and the considerably positive coefficient still indicates a positive relationship.


\begin{figure}
    \centering
    \makebox[\textwidth][c]{\includegraphics[width=1\columnwidth]{Figures/causal_curves1}}
    
    \decoRule
    \caption[Response curves for grants and loans]{Estimated Dose Response Functions, for liquidity (cash) from grants 2021 (top) and loans 2020 (bottom) in relative (left) and absolte (right) terms with 95\% Confidence Bands. For Binomial Distributed Data. The estimate of absolute variants used a lognormal GLM for the GPS estimation due to the different distribution of the variables.}
    \label{fig:Curve1}
\end{figure}


\subsection{The relationship of solvency and aid}


Figure \ref{fig:Curve2} shows the estimated dose response curves for loans and grants in 2021 with their effect on the indebtedness represented by the debt-to-asset ratio.
The downward trend on the left side indicated a negative effect on the debt ratio like the DiD estimates (- 0.03 \%) and an opposing trend in the respective relationship of loans, which confirms the initial assumption on the solvency effects.




\begin{figure}
    \centering
    \makebox[\textwidth][c]{\includegraphics[width=1\columnwidth]{Figures/causal_curves2.png}}
    
    \decoRule
    \caption[Response curves for indebtedness through aid]{Estimated Dose Response Functions, for the debt-to-asset ratio from grants 2021 and loans 2021 in relative terms with 95\% Confidence Bands}
    \label{fig:Curve2}
\end{figure}


Figure \ref{fig:Curve3} show response curves for four selected industries that were used as example cases by \parencite{bischof_bedeutung_2021} for their assumption of heterogenous effects of grants between industries with different cost structures.


Of the four shows relations, the food and beverage sector has the highest transmission according to \parencite{bischof_bedeutung_2021}, followed by the travel agency and tour operator sector and then the creative, arts, and entertainment industry. Last, with the lowest factor is the industry of sports activities, amusement, and recreation.

With the estimated curves in Figure \ref{fig:Curve3}, the food and beverage sector has the most narrow confidence bands and the clearest upwards trends, while the least positive effect on the cash ratio can be interpreted for the travel agency and tour operator sector. However, the curves are highly sensitive to specifications and the wavy and wide confidence bands are indicating limited statistical significance and thus do not allow a reliable judgment confirming or denying a connection between the observed effects and the transmission factors from \parencite{bischof_bedeutung_2021}. 

Considering the additional industry curves from Figure \ref{fig:Curve4} in Appendix 1, it can only be cautiously conclude that the general assumption of heterogeneous effects between industries cannot be rejected with the results obtained.




\begin{figure}
    \centering
    \makebox[\textwidth][c]{\includegraphics[width=1\columnwidth]{Figures/causal_curves_industries1.png}}
    
    \decoRule
    \caption[Response curves for liquidity through aid - by sectors 1]{Estimated Dose Response Functions, for the cash ratio from grants 2021 in selected industries in relative terms with 95\% Confidence Bands}
    \label{fig:Curve3}
\end{figure}


