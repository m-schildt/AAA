% Chapter Template

\chapter{Data Sources} % Main chapter title

\label{Chapter3} % Change X to a consecutive number; for referencing this chapter elsewhere, use \ref{ChapterX}

%----------------------------------------------------------------------------------------
%	SECTION 1
%----------------------------------------------------------------------------------------

\section{Data on Government support}

The thesis uses data from state aid transparency database by the EU COM. The data base contains information about individual award data like beneficiary name, amount, Date of Granting, and the purpose of the state aid \parencite{eu_com_state_2023}. The legal base for the transparency requirement aid payments is Temporary Framework for State aid, however payments under 100.000 EUR (10.000 EUR for agricultural firm) are exempted from the transparency requirement, insofar the data base is not comprehensive. Nevertheless, as of spring 2023 for Germany 135.478 cases of aid related to the COVID-19 pandemic were disclosed under the objective “Remedy for a serious disturbance in the economy”. Unfortunately, the disclosed titles of the aid measures and case numbers doesn’t allow for reconciliation to the official names of the aid programs due to amendments and overlaps.





%----------------------------------------------------------------------------------------
%	SECTION 2
%----------------------------------------------------------------------------------------

\section{Company level financial information}

Policy intervention to "prevent” the effects and save businesses for a fast economic recovery.
First assessments were modeling approaches.
Already lots of early assessments of state aid, also at a firm level.
For getting a better understanding on the effect of aid schemes in Germany a paper analyses the effect of a company’s cost structure on the effectives of aid measures (Bischof, Karlsson, Rostam-Aschar, Simon, 2021). 
This paper assumes that companies within the same sector have a similar cost structure. 
Since aid in Germany is based on the cost structure of companies, the authors conclude that based on the generalized approach of aid schemes, the effectiveness of aid is varying between business sectors.